\chapter{The Birch and Swinnerton-Dyer Formula}

\section{Galois Cohomology}
Galois cohomology is the basic language used for much research into
algebraic aspects of the BSD conjecture.  It was introduced by Lang
and Tate in 1958 in \cite{lang-tate}.  This section contains a survey
of the basic facts we will need in order to define Shafarevich-Tate
groups, discuss descent, and construct Kolyvagin's cohomology classes.

The best basic reference on Galois cohomology is chapters VII and X of
Serre's {\em Local Fields} \cite{serre:localfields} or the (very
similar!) article by Atiyah and Wall in Cassels-Frohlich
\cite[Ch.~IV]{cassels-frohlich}.  See also the article by Gruenberg in
\cite[Ch.~V]{cassels-frohlich} for an introduction to profinite groups such
as $\Gal(\Qbar/\Q)$.  Since this section is only a survey, you should
read one of the above two references in detail, if you haven't
already.  You might also want to read Chapter 1 of
\cite{coates-sujatha} by Coates and Sujatha, which contains an
excellent summary of more advanced topics in Galois cohomology, and
Serre's book {\em Galois Cohomology} \cite{serre:gc} discusses many
general advanced topics in depth.  The original article
\cite{lang-tate} is also well worth reading. 

\subsection{Group Cohomology}
If $G$ is a multiplicative group, the \defn{group ring} $\ZZ[G]$ is
the ring of all finite formal sums of elements of~$G$, with
multiplication defined using distributivity and extending linearly.
Let $A$ be an additive group.  We say that $A$ is a \defn{$G$-module}
if $A$ is equipped with a module structure over the group ring
$\ZZ[G]$.

Let $A^G$ be the submodule of elements of $A$ that are fixed by $G$.
Notice that if $A\to B$ is a homomorphism of $G$-modules, then
restriction defines a homomorphism $A^G \to B^G$, so $A\mapsto A^G$
is a {\em functor}.  In fact, it is a {\em left-exact} functor:
\begin{proposition}
  If $ 0 \to A \to B \to C$ is an exact sequence of $G$ modules, then
  $0 \to A^G \to B^G \to C^G$ is also exact.
\end{proposition}

\begin{definition}[Group Cohomology]
The \defn{group cohomology $H^n(G,A)$} is by definition the
{\em right derived functors} of the left exact functor $A \to A^G$.
These are the unique, up to canonical equivalence, functors $H^n$
such that
\begin{itemize}
\item The sequence
$$
 0 \to A^G \to B^G \to C^G \xrightarrow{\delta} 
H^1(G,A) \to \cdots \to H^n(G,A) \to H^n(G,B) \to H^n(G,C) \xrightarrow{\delta} H^{n+1}(G,A) \to \cdots
$$
is exact. 
\item If $A$ is \defn{coinduced}, i.e., $A = \Hom(\ZZ[G], X)$ for $X$ an abelian group, then
$$
  \H^n(G,A) = 0 \text{ for all $n\geq 1$}.
$$
\end{itemize}
\end{definition}

\begin{remark}
For those familiar with the $\Ext$ functor, we have
$$
\H^n(G,A) = \Ext_{\ZZ[G]}^n(\ZZ,A).
$$
\end{remark}

We construct $\H^n(G,A)$ explicitly as follows.
Consider $\ZZ$ as a $G$-module, equipped with the
trivial $G$-action.
Consider the following free resolution of $\ZZ$.
Let $P_i$ be the free $\ZZ$-module with basis
the set of $i+1$ tuples
$(g_0,\ldots, g_i) \in G^{i+1}$, and
with $G$ acting on $P_i$ componentwise:
$$
   s (g_0,\ldots, g_i) = (sg_0, \ldots, sg_i).
$$
The homomorphism $d:P_i\to P_{i+1}$ is given
by 
$$
 d(g_0,\ldots, g_i) = \sum_{j=0}^{i} (-1)^j
       (g_0,\ldots,g_{j-1},g_{j+1},\ldots g_i),
$$
and $P_0\to \ZZ$ is given by sending every
element $(g_0)$ to $1\in\Z$. 

The cohomology groups $\H^i(G,A)$ are then
the cohomology groups of the complex
$K_i = \Hom_{\ZZ[G]}(P_i, A)$. 
We identify an element of $K_i$
with a function $f:G^{i+1} \to A$
such that the condition
$$
  f(sg_0,\ldots, sg_i) = s f(g_0,\ldots, g_i)
$$
holds. 
Notice that such an $f\in K_i$ is uniquely determined
by the function (of $i$ inputs)
$$
 \vphi(g_1,\ldots, g_i) = f(1,g_1,g_1 g_2, \ldots, g_1\cdots g_i).
$$

The boundary map $d:K_i\to K_{i+1}$
on such functions $\vphi \in K_{i}$
is then given explicitly by the formula
\begin{align*}
 (d\vphi)(g_1,\ldots, g_{i+1})
    &= g_1 \vphi(g_2,\ldots, g_{i+1}) 
   + \sum_{j=1}^{i} (-1)^j \vphi(g_2,\ldots, g_{j} g_{j+1}, \ldots, g_{i+1})\\
   &+ (-1)^{i+1} \vphi(g_1,\ldots, g_{i}).
 \end{align*}
The group of 
\defn{$n$-cocycles} is the group of $\vphi \in K_{n}$, 
as above are functions of $n$ variables such that
$d\vphi = 0$.
The subgroup of \defn{$n$-coboundaries}
is the image of $K_{n+1}$ 
under $d$.
Explicitly, the cohomology group $H^n(G,A)$ is 
the quotient of the group group of $n$-cocycles
modulo the subgroup of $n$-coboundaries. 

When $n=1$, the $1$-cocycles
are the maps $G\to A$ such that
$$
 \vphi(g g') = g\vphi(g') + \vphi(g),
$$
and $\vphi$ is a coboundary if there exists $a\in A$
such that $\vphi(g) = ga - a$ for all $g \in G$.
Notice that if $G$ acts trivially on $A$, then
$$
  \H^1(G,A) = \Hom(G,A).
$$

\subsection{The inf-res Sequence}\label{sec:infres}
Suppose $G$ is a group and $H$ is a normal subgroup
of $G$, and $A$ is a $G$-module.  
Then for any $n\geq 0$,
there are natural homomorphisms
$$
  \res : \H^n(G,A) \to \H^n(H,A)
$$
and 
$$
  \inf : \H^n(G/H,A^H) \to \H^n(G,A)
$$
Require that we view $n$-cocycles as
certain maps on the $n$-fold product 
of the group.
On cocycles, the map $\res$ is
obtained by simply restricting
a cocycle, which is a map $G^i\to A$,
to a map $H^i\to A$.
The second map $\inf$ is obtained
by precomposing a cocycle
$(G/H)^i\to A^H$ with the natural
map $G^i \to (G/H)^i$.

\begin{proposition}
The \defn{inf-res sequence}
$$
 0 \to \H^1(G/H,A^H) \xrightarrow{\inf} \H^1(G,A)
 \xrightarrow{\res} \H^1(H,A)
$$
is exact.
\end{proposition}
\begin{proof}
See \cite[\S{}VII.6]{serre:localfields}.
\end{proof}

\subsection{Galois Cohomology}
Let $K$ be a field and $L$ a finite \defn{Galois extension} of 
$K$, so the set of field automorphisms of $L$
that fix $K$ equals the dimension of $L$ viewed
as a $K$-vector space. 

For any $\Gal(L/K)$-module $A$ and any $n\geq 0$, let 
$$
 \H^n(L/K, A) = \H^n(\Gal(L/K), A).
$$
If $M/L/K$ is a tower of Galois extensions of $K$
and suppose $\Gal(M/K)$ acts on $A$.
Then inf defines a map
\begin{equation}\label{eqn:infgc}
\H^n(L/K,A^L) \to \H^n(M/K,A).
\end{equation}


Let $K^{\sep}$ denote a separable closure of $K$ and 
suppose $A$ is a (continuous) $\Gal(K^{\sep}/K)$-module.
(Note -- if $K$ has characteristic $0$, then a separable
closure is the same thing as an algebraic closure.)
For any subfield $L\subset K^{\sep}$ that contains $K$,
let $A(L) = A^L$. 
Let
$$
  \H^n(K,A) = \varinjlim_{L/K\text{ finite Galois}} \H^n(L/K, A(L)),
$$
where the direct limit is with respect to the
maps \eqref{eqn:infgc}.  We can think of this direct
limit as simply the union of all the groups, where
we identify two elements if they are eventually
equal under some map \eqref{eqn:infgc}.

One can prove (see \cite[Ch.~V]{cassels-frohlich}) that changing the
choice of separable closure $K^{\sep}$ only changes $\H^n(K,A)$ by
unique isomorphism, i.e., the construction is essentially independent
of the choice of seperable closure.



\newpage
\section{The Shafarevich-Tate Group}\label{sec:sha}

In this section we discuss Galois cohomology of elliptic
curves, introduce the Kummmer sequence, define the
Selmer group, the Shafarevich-Tate group and dicuss
descent and the Mordell-Weil theorem. 

\subsection{The Elliptic Curve Kummer Sequence}\label{sec:kummer}

Let $E$ be an elliptic curve over a number field $K$.  Consider the
abelian group $E(\Qbar)$ of all points on $E$ defined over a fixed
choice $\Qbar$ of algebraic closure of $\QQ$.  Then $A$ is a module
over $\Gal(\Qbar/K)$, and we may consider the Galois cohomology
groups
$$
  \H^n(K, E),\qquad\text{for $n=0,1,2,\ldots$}
$$
which are of great interest in the study of elliptic curves,
especially for $n=0,1$.

If $L$ is a finite Galois extension of $K$,
then the inf-res sequence, written in terms
of Galois chomology, is
$$
0 \to \H^1(L/K, E(L)) \to \H^1(K, E) \to \H^1(L, E).
$$

For any positive integer $n$ consider the homomorphism
$$
 [n] : E(\Qbar) \to E(\Qbar).
$$
This is a surjective homomorphism of abelian groups,
so we have an exact sequence
$$
  0 \to E[n] \to E \xrightarrow{[n]} E \to 0.
$$
The associated long exact sequence of Galois cohomology
is
$$
 0 \to E(K)[n] \to E(K) \xrightarrow{[n]} E(K)
\to \H^1(K, E[n]) \to \H^1(K, E)\xrightarrow{[n]}
\H^1(K, E)\to \cdots.
$$
An interesting way to rewrite the begining part
of this sequence is as
\begin{equation}\label{eqn:eckummer}
 0 \to E(K)/n E(K) \to \H^1(K,E[n]) \to H^1(K,E)[n]\to 0.
\end{equation}
The sequence \eqref{eqn:eckummer} is called the \defn{Kummer sequence}
associated to the elliptic curve. 

\subsection{The Global-to-Local Restriction Maps}
Let $\wp$ be a prime ideal of the ring $\O_K$ of integers
of the number field $K$, and let $K_\wp$ be the completion
of $K$ with respect to $\wp$.   Thus $K_{\wp}$ is a finite
extension the field $\Q_p$ of $p$-adic numbers.  

More explicitly, if $K=\QQ(\alpha)$, with $\alpha$ a root of the
irreducible polynomial $f(x)$, then the prime ideals $\wp$ correspond
to the irreducible factors of $f(x)$ in $\ZZ_p[x]$. The fields
$K_{\wp}$ then correspond to adjoing roots of each of these
irreducible factors of $f(x)$ in $\Z_p[x]$.  Note that for most $p$, a
generalization of Hensel's lemma (see Section~\ref{sec:padicprelim})
asserts that we can factor $f(x)$ by factoring $f(x)$ modulo $p$ and
iteratively lifting the factorization.

We have a natural map $\Gal(\Qpbar/K_\wp) \to \Gal(\Qbar/K)$
got by restriction; implicit in this is a {\em choice} of
embedding of $\Qbar$ in $\Qpbar$ that sends $K$ into $K_v$.
We may thus view $\Gal(\Qpbar/K_{\wp})$ as a subgroup
of $\Gal(\Qbar/K)$. 

Let $A$ be any $\Gal(\Qbar/K)$ module.  Then
this restriction map induces a restriction map on Galois cohomology
$$
\res_{\wp} : \H^1(K, A) \to \H^1(K_{\wp}, A).
$$
Recall that in terms of $1$-cocycles this sends
a set-theoretic map (a crossed-homomorphism)
$f:\Gal(\Qbar/K)\to A$ to a map
$\res_{\wp}(f):\Gal(\Qpbar/K_{\wp}) \to A$.


Likewise there is a restriction map for each
real Archimedian prime $v$, i.e., for each embedding 
$K\to \RR$ we have a map
$$
\res_v : \H^1(K,A) \to \H^1(\R, A).
$$

\begin{exercise}
  Let $A=E(\CC)$ be the group of points on an elliptic curve over
  $\RR$.  Prove that $\H^1(\RR,E) = \H^1(\CC/\RR,E(\CC))$ is a group
  of order $1$ or $2$.
\end{exercise}

\begin{exercise}
  Prove that for any Galois moduloe $A$ and for all primes $\wp$ the
  kernel of $\res_{\wp}$ does not depend on the choice of embedding of
  $\Qbar$ in $\Qpbar$.  (See \cite[Ch.~V]{cassels-frohlich}).
\end{exercise}

\subsection{The Selmer Group}
Let $E$ be an elliptic curve over a number field $K$.
Let $v$ be either a prime $\wp$ of $K$ or a real
Archimedian place (i.e., embedding $K\to \RR$).
As in Section~\ref{sec:kummer} we also obtain
a local Kummer sequence 
$$
  0 \to E(K_v)/n E(K_v) \to \H^1(K_v,E[n]) \to H^1(K_v,E)[n]\to 0.
$$
Putting these together for all $v$ we obtain a commutative
diagram:
\begin{equation}\label{eqn:lg1}
\xymatrix@=1.2em{
 0 \ar[r]& {E(K)/n E(K)} \ar[r]\ar[d]& {\H^1(K,E[n])} \ar[r]\ar[d]
      &  {H^1(K,E)[n]} \ar[r]\ar[d]& 0 \\
  0 \ar[r]& {\prod_v E(K_v)/n E(K_v)} \ar[r]& {\prod_v \H^1(K_v,E[n])}
 \ar[r]& {\prod_v H^1(K_v,E)[n]} \ar[r]& 0.
}
\end{equation}
\begin{definition}
The \defn{$n$-Selmer group} of an elliptic curve $E$ over a number
field $K$ is
$$
 \Sel^{(n)}(E/K) = \ker\left(\H^1(K,E[n]) \to \prod_v \H^1(K_v,E)[n]\right).
$$
\end{definition}


\subsection{The Shafarevich-Tate Group and the Mordell-Weil Theorem}
\label{sec:shadef}
\begin{definition}[Shafarevich-Tate Group]
The \defn{Shafarevich-Tate group} of an elliptic curve $E$ over a number
field $K$ is
$$
 \Sha(E/K) = \ker\left(\H^1(K,E) \to \prod_v \H^1(K_v,E)\right).
$$
\end{definition}

For any positive integer $n$,
we may thus add in a row to (\ref{eqn:lg1}):
$$
\xymatrix@=1.2em{
 0 \ar[r]& {E(K)/n E(K)} \ar[r]\ar@{=}[d] & \Sel^{(n)}(E/K)\ar[r]\ar@{^(->}[d] 
        & \Sha(E/K)[n] \ar[r]\ar@{^(->}[d] & 0\\
 0 \ar[r]& {E(K)/n E(K)} \ar[r]\ar[d]& {\H^1(K,E[n])} \ar[r]\ar[d]
      &  {H^1(K,E)[n]} \ar[r]\ar[d]& 0 \\
  0 \ar[r]& {\prod_v E(K_v)/n E(K_v)} \ar[r]& {\prod_v \H^1(K_v,E[n])}
 \ar[r]& {\prod_v H^1(K_v,E)[n]} \ar[r]& 0.
}
$$

The \defn{$n$-descent sequence} for $E$ is the short exact sequence
\begin{equation}\label{eqn:ndescent}
  0 \to E(K)/n E(K) \to \Sel^{(n)}(E/K) \to \Sha(E/K)[n] \to 0.
\end{equation}


\begin{theorem}\label{thm:selmerfinite}
 For every integer $n$ the group $\Sel^{(n)}(E/K)$ is finite. 
\end{theorem}
\begin{proof}[Sketch of Proof]
Let $K(E[n])$ denote the finite Galois extension of $K$
obtained by adjoining to $K$ all $x$ and $y$ coordinates
of elements of $E(\Qbar)$ of order dividing $n$.
The inf-res sequence for $K(E[n])/K$ is 
\begin{equation}\label{eqn:infresken}
  0 \to \H^1(K(E[n])/K, E[n]) \to \H^1(K, E[n]) \to \H^1(K(E[n]), E[n]).
\end{equation}
Because $\Gal(K(E[n])/K)$ and $E[n]$ are both finite groups,
the cohomology group $\H^1(K(E[n])/K, E[n])$ is also finite.

Since $\Sel^{(n)}(E/K) \subset \H^1(K,E[n])$, 
restriction defines a map 
\begin{equation}\label{eqn:selmerrestrict}
  \Sel^{(n)}(E/K) \to \Sel^{(n)}(E/K[n]).
\end{equation}
The kernel of \eqref{eqn:selmerrestrict}
is finite since it 
is contained in the first term of \eqref{eqn:infresken},
which is finite.   It thus suffices to prove
that $\Sel^{(n)}(E/K[n])$ is finite.

But 
$$
  \Sel^{(n)}(E/K[n]) \subset \H^1(K[n], E[n]) \isom \Hom(\Gal(\Qbar/K[n]), E[n])).
$$
So each element of $\Sel^{(n)}(E/K[n])$ determines (and is determined by)
a homomorphism $\Gal(\Qbar/K[n]) \to (\Z/n\Z)^2$.
That that the fixed field of such a homomorphism is
a Galois extension of $K[n]$ with Galois group
contained in $(\Z/n\Z)^2$.

To complete the proof, one uses the theory of
elliptic curves over local fields to show that
there is a finite set $S$ of primes such that
any such homomorphism corresponding to an element of 
the Selmer group corresponds to an extension 
of $K[n]$ ramified only at primes in $S$.  Then the two
main theorems of algebraic number theory --- that
class groups are finite and unit groups are finitely
generated --- together imply that there are only
finitely many such extensions of $K[n]$.

% Let $L$ be the extension of $K[n]$ obtain by adjoining
% to $K[n]$ the coordinates of all $n$-th roots of
% elements of $E(K[n])$, i.e., for every $P \in E(K[n])$
% adjoing coordinates of all points $Q \in E(\Qbar)$
% such that $nQ = P$.   A study of elliptic curves over
% local fields implies that $L$ is only ramified at a
% finite set $S$ of primes (see \cite[Prop.~VIII.1.5]{silverman:aec}).
% On the other hand, the Galois group of $L$ has
% exponent $n$, since $L$ is just the compositum of all
% splitting fields of elements of $

\end{proof}


\begin{exercise}
  Prove the that $E[n]$ is a finite Galois extension of $K$.
\end{exercise}


\begin{theorem}[Mordell-Weil]
 The group $E(\QQ)$ is finitely generated.
\end{theorem}
\begin{proof}
The exact sequence \eqref{eqn:ndescent} with $n=2$
and Theorem~\ref{thm:selmerfinite} imply that
$E(\QQ)/ 2 E(\QQ)$ is a finite group.
Recall Lemma~\ref{lem:allgen} which asserted
that if $B$ is a positive real number such that
$$
   S = \{ P \in E(\QQ)\, : \, \hat{h}(P) \leq B \}
$$
contains a set of generators for $E(\QQ)/2 E(\QQ)$, then
$S$ generates $E(\QQ)$.  Since $E(\QQ)/2 E(\QQ)$
is finite, it makes sense to define $B$ to 
be the maximum of the heights of arbitrary
lifts of all the elements of $E(\QQ)/2 E(\QQ)$.
Then the corresponding set $S$ generates
$E(\QQ)$.  A basic fact about heights is that
the set of points of bounded height is finite,
i.e., $S$ is finite, so $E(\QQ)$ is finitely generated.
\end{proof}

\subsection{Some Conjectures and Theorems about the Shafarevich-Tate Group}
\begin{conjecture}[Shafarevich-Tate]\label{conj:shafinite}
Let $E$ be an elliptic curve over a number field $K$.
Then the group $\Sha(E/K)$ is finite.
\end{conjecture}

\begin{theorem}[Rubin]
If $E$ is a CM elliptic curve over $\QQ$ with $L(E,1)\neq 0$,
then $\Sha(E/\QQ)$ is finite.  (He proved more than just this.)
\end{theorem} 
Thus Rubin's theorem proves that the Shafarevich-Tate group
of the CM elliptic curve $y^2 + y = x^3 - 7$ of conductor $27$
is finite. 

\begin{theorem}[Kolyvagin et al.]
If $E$ is an elliptic curve over $\QQ$ with $\ord_{s=1} L(E,s) \leq 1$,
then $\Sha(E/\QQ)$ is finite. 
\end{theorem}
Kolyvagin's theorem is proved in a completely different way than
Rubin's theorem.  It combines the Gross-Zagier theorem, the modularity
theorem that there is a map $X_0(N)\to E$, a nonvanishing result about
the special values $L(E^D,1)$ of quadratic twists of $E$, and a highly
original explicit study of the structure of the images of certain
points on $X_0(N)(\Qbar)$ in $E(\Qbar)$.

\begin{theorem}[Cassels]\label{thm:casselspairing}
Let $E$ be an elliptic curve over a number field $K$.
There is an alternating pairing on $\Sha(E/K)$, which
is nondegenerate on the quotient of 
$\Sha(E/K)$ by its maximal divisible subgroup.  Moreover,
if $\Sha(E/K)$ is finite then $\#\Sha(E/K)$ is a perfect
square. 
\end{theorem}

For an abelian group $A$ and a prime $p$, let
$A(p)$ denote the subgroup of elements of $p$ power order in $A$.

The following problem remains open.  It helps illustrate
our ignorance about Conjecture~\ref{conj:shafinite}
in any cases beyond those mentioned above.
\begin{problem}
Show that there is an elliptic curve $E$ over $\QQ$
with rank $\geq 2$ such that $\Sha(E/\QQ)(p)$ is
finite for infinitely many primes $p$.
\end{problem}


\section{The Birch and Swinnerton-Dyer Formula}
\begin{quotation}
  ``The subject of this lecture is rather a special one.  I want to
  describe some computations undertaken by myself and Swinnerton-Dyer
  on EDSAC, by which we have calculated the zeta-functions of certain
  elliptic curves.  As a result of these computations we have found an
  analogue for an elliptic curve of the Tamagawa number of an
  algebraic group; and conjectures have proliferated. [$\ldots$] I
  would like to stress that though the associated theory is both
  abstract and technically complicated, the objects about which I
  intend to talk are usually simply defined and often machine
  computable; experimentally we have detected certain relations
  between different invariants, but we have been unable to approach
  proofs of these relations, which must lie very deep.''\\
\mbox{}\hspace{20em}--\,\,Bryan Birch
\end{quotation}


\begin{conjecture}[Birch and Swinnerton-Dyer]\label{conj:bsdf}
Let $E$ be an elliptic curve over $\QQ$ of rank $r$.  
Then $r = \ord_{s=1} L(E,s)$ and 
\begin{equation}\label{eq:bsd}
 \frac{L^{(r)}(E,1)}{r!} = \frac{\Omega_E \cdot \Reg(E) \cdot \#\Sha(E/\Q) \cdot \prod_{p} c_p }{\#E(\Q)_{\tor}^2}.
\end{equation}
\end{conjecture}


Let 
\begin{equation}\label{eq:bsdweq}
  y^2 + \ua_1 xy + \ua_3 y = x^3 +\ua_2 x^2 + \ua_4 x + \ua_6
\end{equation}
be a minimal Weierstrass equation for $E$.

Recall from Section~\ref{sec:modsymmeasure} that
the \defn{real period} $\Omega_E$ is the integral 
$$
\Omega_E = \int_{E(\RR)} \frac{dx}{2y + \ua_1 x + \ua_3}.
$$
See \cite[\S3.7]{cremona:algs} for an explanation about
how to use the Gauss arithmetic-geometry mean to 
efficiently compute $\Omega_E$.

To define the \defn{regulator} $\Reg(E)$ let $P_1,\ldots, P_n$
be a basis for $E(\QQ)$ modulo torsion and recall
the N\'eron-Tate canonical height pairing $\langle\,,\, \rangle$ from
Section~\ref{sec:bsdimpliescomputable}.
The real number $\Reg(E)$ is the absolute value of
the determinant of the $n\times n$ matrix whose $(i,j)$
entry is $\langle P_i, P_j \rangle$.   See 
\cite[\S3.4]{cremona:algs} for a discussion of how to
compute $\Reg(E)$.

We defined the group $\Sha(E/\QQ)$ in Section~\ref{sec:shadef}.  In
general it is not known to be finite, which led to Tate's famous
assertion that the above conjecture ``relates the value of a function
at a point at which it is not known to be defined\footnote{When $E$ is
  defined over $\QQ$ it is now known that $L(E,s)$ is defined
  overwhere.} to the order of a group that is not known to be
finite.''  The paper \cite{bsdalg1} discusses methods for computing
$\#\Sha(E/\QQ)$ in practice, though no general algorithm for
computing $\#\Sha(E/\QQ)$ is known.  In fact, in general even if
we assume truth of the BSD rank conjecture (Conjecture~\ref{conj:bsdrank})
and assume that $\Sha(E/\QQ)$ is finite,
there is still no known way to compute $\#\Sha(E/\QQ)$, i.e.,
there is no analogue of Proposition~\ref{prop:bsdalgrank}.
Given finiteness of $\Sha(E/\QQ)$ we can compute
the $p$-part $\Sha(E/\QQ)(p)$ of $\Sha(E/\QQ)$ for any prime
$p$, but we don't know when to stop considering new primes~$p$. 
(Note that when $r_{E,\an}\leq 1$, Kolyvagin's work provides
an explicit upper bound on $\#\Sha(E/\QQ)$, so 
in that case $\Sha(E/\QQ)$ is computable.)



The \defn{Tamagawa numbers} $c_p$ are $1$ for all primes $p\nmid
\Delta_E$, where $\Delta_E$ is the discriminant of \eqref{eq:bsdweq}.
When $p\mid \Delta_E$, the number $c_p$ is a more refined measure
of the structure of the $E$ locally at $p$.  If $p$ is a prime
of {\em additive reduction} (see Section~\ref{sec:complexles}), then
one can prove that $c_p \leq 4$.   The other alternatives are
that $p$ is a prime of split or nonsplit multiplicative reduction.
If $p$ is a {\em nonsplit prime}, then 
$$
  c_p = \begin{cases} 1 & \text{ if $\ord_p(\Delta)$ is odd} \\
                      2 & \text{ otherwise}
\end{cases}
$$  
If $p$ is a prime
of {\em split multiplicative} reduction then 
$$
 c_p = \ord_p(\Delta)
$$
can be arbitrarily large.  The above discussion completely determines
$c_p$ except when $p$ is an additive prime -- see \cite[\S3.2]{cremona:algs}
for a discussion of how to compute $c_p$ in general.

For those that are very familiar with elliptic curves
over local fields, 
$$
 c_p = [E(\QQ_p) : E^0(\QQ_p)],
$$
where $E^0(\QQ_p)$ is the subgroup of $E(\QQ_p)$ of points
that have nonsingular reduction modulo $p$.

For those with more geometric background, we offer the following
conceptual definition of $c_p$.  Let $\cE$ be the \defn{N\'eron model}
of $E$. This is the unique, up to unique isomorphism, smooth
commutative (but not proper!) group scheme over $\ZZ$ that has
generic fiber $E$ and satisfies the N\'eron mapping property:
\begin{quote}
for any smooth group scheme $X$ over $\Z$ the natural map
$$\Hom(X, \cE) \to \Hom(X_\QQ, E)$$
is an isomorphism.
\end{quote}
In particular, note that $\cE(\ZZ) \isom E(\QQ)$.
For each prime $p$, the reduction $\cE_{\F_p}$ of the N\'eron 
model modulo $p$ is a smooth commutative group
scheme over $\F_p$ (smoothness is a property of morphisms
that is closed under base extension).  Let $\cE_{\F_p}^0$
be the identity component of the group scheme $\cE_{\F_p}$,
i.e., the connected component of $\cE_{\F_p}^0$ that contains
the $0$ section.  The \defn{component group} of $E$ at $p$
is the quotient group scheme
$$
  \Phi_{E,p} = \cE_{\F_p} / \cE_{\F_p}^0,
$$
which is a finite \'etale group scheme over $\F_p$.
Finally
$$
  c_p = \# \Phi_{E,p}(\F_p).
$$


\section{Examples: The Birch and Swinnerton-Dyer Formula}

In each example below we use \sage to compute the conjectural order
of $\Sha(E/\QQ)$ and find that it appears to be
the square of an integer as predicted by 
Theorem~\ref{thm:casselspairing}.

\subsection{Example: A Curve of Rank 0}\label{sec:ex11a}
Consider the elliptic curve $E$ with Cremona label 11a,
which is one the 3 curves of smallest conductor.
We now compute each of the quantities
in  Conjecture~\ref{conj:bsdf}.
First we define the curve $E$ in \sage and compute its
rank:
\begin{verbatim}
sage: E = EllipticCurve('11a'); E
Elliptic Curve defined by y^2 + y = x^3 - x^2 - 10*x - 20 
over Rational Field
sage: E.rank()
0
\end{verbatim}%link

\noindent{}Next we compute the number $L(E,1)$ to double precision (as an element
of the real double field {\tt RDF}):
%link
\begin{verbatim}
sage: L = RDF(E.Lseries(1)); L
0.253841860856
\end{verbatim}%link

\noindent{}We next compute the real period:
%link
\begin{verbatim}
sage: Om = RDF(E.omega()); Om
1.26920930428
\end{verbatim}%link

\noindent{}To compute $\prod c_p$ we factor
the discriminant of $E$.  It turns at that only $11$
divides the discriminant, and since the reduction
at $11$ is split multiplicative the Tamagawa number
is $5=\ord_{11}(\Delta_E)$. 
%link
\begin{verbatim}
sage: factor(discriminant(E))
-1 * 11^5
sage: c11 = E.tamagawa_number(11); c11
5
\end{verbatim}%link

\noindent{}Next we compute the regulator, which is $1$ since $E$
rank $0$.
%link
\begin{verbatim}
sage: Reg = RDF(E.regulator()); Reg
1.0
\end{verbatim}%link

\noindent{}The torsion subgroup has order $5$.
%link
\begin{verbatim}
sage: T = E.torsion_order(); T
5
\end{verbatim}%link

\noindent{}Putting everything together in \eqref{eq:bsd} and
solving for the conjectural order of $\Sha(E/\QQ)$, we
see that Conjecture~\ref{conj:bsdf} for $E$ is equivalent
to the assertion that $\Sha(E/\QQ)$ has order $1$.
%link
\begin{verbatim}
sage: Sha_conj = L * T^2 / (Om * Reg * c11); Sha_conj
1.0
\end{verbatim}

\subsection{Example: A Rank 0 curve with nontrivial Sha}
Consider the curve $E$ with label 681b.  This curve
has rank $0$, and we compute the conjectural order of $\#\Sha(E/\QQ)$
as in the previous section:
\begin{verbatim}
sage: E = EllipticCurve('681b'); E
Elliptic Curve defined by y^2 + x*y  = x^3 + x^2 - 1154*x - 15345 
over Rational Field
sage: E.rank()
0
sage: L = RDF(E.Lseries(1)); L
1.84481520613
sage: Om = RDF(E.omega()); Om
0.81991786939
\end{verbatim}%link

\noindent{}There are two primes of bad reduction this time.
%link
\begin{verbatim}
sage: factor(681)
3 * 227
sage: factor(discriminant(E))
3^10 * 227^2
sage: c3 = E.tamagawa_number(3); c227 = E.tamagawa_number(227)
sage: c3, c227
(2, 2)
sage: Reg = RDF(E.regulator()); Reg
1.0
sage: T = E.torsion_order(); T
4
\end{verbatim}%link

\noindent{}In this case it turns out that
$\#\Sha(E/\QQ)$ is conjecturally $9$.
%link
\begin{verbatim}
sage: Sha_conj = L * T^2 / (Om * Reg * c3*c227); Sha_conj
9.0
\end{verbatim}

\subsection{Example: A Curve of Rank 1}\label{sec:ex37a}
Let $E$ be the elliptic curve with label 37a,
which is
the curve of rank $1$ with smallest conductor. 
We define $E$ and compute its rank, which is $1$.
\begin{verbatim}
sage: E = EllipticCurve('37a'); E
Elliptic Curve defined by y^2 + y = x^3 - x over 
Rational Field
sage: E.rank()
1
\end{verbatim}%link

\par\noindent{}We next compute the value $L'(E,1)$. The
corresponding function in \sage takes a bound on the number
of terms of the $L$-series to use, and returns an approximate
to $L'(E,1)$ along with a bound on the error (coming from
the tail end of the series). 
%link
\begin{verbatim}
sage: L, error = E.Lseries_deriv_at1(200); L, error
(0.305999773834879, 2.10219814818300e-90)
sage: L = RDF(L); L
0.305999773835
\end{verbatim}%link

\par\noindent{}We compute $\Omega_E$ and the Tamagawa number,
regulator, and torsion as above.
%link
\begin{verbatim}
sage: Om = RDF(E.omega()); Om
5.98691729246
sage: factor(discriminant(E))
37
sage: c37 = 1
sage: Reg = RDF(E.regulator()); Reg
0.05111140824
sage: T = E.torsion_order(); T
1
\end{verbatim}%link

\par\noindent{}Finally, we solve and find that
the conjectural order of $\Sha(E/\Q)$ is $1$.
%link
\begin{verbatim}
sage: Sha_conj = L * T^2 / (Om * Reg * c37); Sha_conj
1.0
\end{verbatim}

\subsection{Example: A curve of rank $2$}
Let $E$ be the elliptic curve 389a of rank $2$, which is
the curve of rank $2$ with smallest conductor. 
\begin{verbatim}
sage: E = EllipticCurve('389a'); E
Elliptic Curve defined by y^2 + y = x^3 + x^2 - 2*x
over Rational Field
sage: E.rank()
2
\end{verbatim}%link

\noindent{}Because the curve has rank $2$, we use Dokchitser's
$L$-function package to approximate $L^{(2)}(E,1)$
to high precision:
%link
\begin{verbatim}
sage: Lser = E.Lseries_dokchitser()
sage: L = RDF(abs(Lser.derivative(1,2))); L
1.51863300058
\end{verbatim}%link

\noindent{}We compute the regulator, Tamagawa numbers, and torsion
as usual:
%link
\begin{verbatim}
sage: Om = RDF(E.omega()); Om
4.98042512171
sage: factor(discriminant(E))
389
sage: c389 = 1
sage: Reg = RDF(E.regulator()); Reg
0.152460177943
sage: T = E.torsion_order(); T
1
\end{verbatim}%link

Finally we solve for the conjectural order of $\#\Sha(E/\QQ)$.
%link
\begin{verbatim}
sage: Sha_conj = (L/2) * T^2 / (Om * Reg * c389)
sage: Sha_conj
1.0
\end{verbatim}

We pause to emphasize that just getting something that looks
like an integer by computing
\begin{equation}\label{eq:sha389}
  \frac{L^{(r)}(E,1)}{r!} \cdot \#E(\QQ)_{\tor}^2 / (\Omega_E \cdot \Reg(E) \cdot \prod c_p)
\end{equation}
is already excellent evidence for Conjecture~\ref{conj:bsdf}.
There is also a subtle and deep open problem here:
\begin{openproblem}
Let $E$ be the elliptic curve 389a above. Prove
that the quantity \eqref{eq:sha389} is a rational number. 
\end{openproblem}
For curves $E$ of analytic rank $0$ it is easy to prove
using modular symbols that the conjectural order of $\Sha(E/\QQ)$
is a rational number.  For curves with analytic rank $1$, this
rationality  follows from the very deep Gross-Zagier theorem.
For curves of analytic rank $\geq 2$ there is not a single 
example in which the conjectural order of $\Sha(E/\QQ)$
is known to be a rational number.


\subsection{Example: A Rank 3 curve}
The curve $E$ with label 5077a has rank $3$.  This 
is the curve with smallest conductor and rank $3$.
\begin{verbatim}
sage: E = EllipticCurve('5077a'); E
Elliptic Curve defined by y^2 + y = x^3 - 7*x + 6 
over Rational Field
sage: E.rank()
3
\end{verbatim}%link

\noindent{}We compute $L(E,s)$ using Dokchitser's algorithm.  Note
that the order of vanishing appears to be $3$.
%link
\begin{verbatim}
sage: E.root_number()
-1
sage: Lser = E.Lseries_dokchitser()
sage: Lser.derivative(1,1)
-5.63436295355925e-22
sage: Lser.derivative(1,2)
2.08600476044634e-21
sage: L = RDF(abs(Lser.derivative(1,3))); L
10.3910994007
\end{verbatim}%link

\noindent{}That the order of vanishing is really $3$ follows from the
Gross-Zagier theorem, which asserts that $L'(E,1)$ is a nonzero
multiple of the N\'eron-Tate canonical height of a certain point on
$E$ called a Heegner point.  One can explicitly construct this
point\footnote{This is not yet implemented in \sage; if it were, there
  would be an example right here.} on $E$ and find that it is torsion,
hence has height $0$, so $L'(E,1)=0$. That $L''(E,1)=0$ then follows
from the functional equation (see Section~\ref{sec:complexles}).
Finally we compute the other BSD invariants:
%link
\begin{verbatim}
sage: Om = RDF(E.omega()); Om
4.15168798309
sage: factor(discriminant(E))
5077
sage: c5077 = 1
sage: Reg = RDF(E.regulator()); Reg
0.417143558758
sage: T = E.torsion_order(); T
1
\end{verbatim}%link

\noindent{}Putting everything together we see that the
conjectural order of $\Sha(E/\QQ)$ is $1$.
%link
\begin{verbatim}
sage: Sha_conj = (L/6) * T^2 / (Om * Reg * c5077)
sage: Sha_conj
1.0
\end{verbatim}
\noindent{}Note that just as was the case with the
curve 389a above, we do not know that the above
conjectural order of $\Sha(E/\QQ)$ is a
rational number, since there are no know theoretical
results that relate any of the three real numbers
$L^{(3)}(E,1)$, $\Reg(E/\QQ)$, and $\Omega_{E/\QQ}$. 

\subsection{Example: A Rank 4 curve}
Let $E$ be the curve of rank $4$ with label 234446b.  It is 
likely that this is the curve with smallest conductor and rank $4$ (a
big calculation of the author et al. shows that there are no rank $4$
curves with smaller {\em prime} conductor).

\begin{verbatim}
sage: E = EllipticCurve([1, -1, 0, -79, 289]); E
Elliptic Curve defined by y^2 + x*y  = x^3 - x^2 - 79*x + 289 
over Rational Field
sage: E.rank()
4
\end{verbatim}%link

We next compute $L(E,1)$, $L'(E,1)$, $L^{(2)}(E,1)$, $L^{(3)}(E,1)$,
and $L^{(4)}(E,1)$.  All these special values {\em look} like they are
$0$, except for $L^{(4)}(E,1)$ which is about $214$, hence clearly
nonzero.  One can prove that $L(E,1)=0$ (e.g., using denominator
bounds coming from modular symbols), hence since the root number is
$+1$, we have either $r_{E,\an} = 2$ or $r_{E,\an}=4$, and of course
suspect (but cannot prove yet) that $r_{E,\an}=4$.
\noindent{}
%link
\begin{verbatim}
sage: E.root_number()
1
sage: Lser = E.Lseries_dokchitser()
sage: Lser(1)
1.43930352980778e-18
sage: Lser.derivative(1,1)
-4.59277879927938e-24
sage: Lser.derivative(1,2)
-8.85707917856308e-22
sage: Lser.derivative(1,3)
1.01437455701212e-20
sage: L = RDF(abs(Lser.derivative(1,4))); L
214.652337502
\end{verbatim}%link

\noindent{}As above, we compute the other BSD invariants of $E$.
%link
\begin{verbatim}
sage: Om = RDF(E.omega()); Om
2.97267184726
sage: factor(discriminant(E))
2^2 * 117223
sage: c2 = 2
sage: c117223 = 1
sage: Reg = RDF(E.regulator()); Reg
1.50434488828
sage: T = E.torsion_order(); T
1
\end{verbatim}%link

Finally, putting everything together, we see that the conjectural
order of $\Sha(E/\QQ)$ is 1.
\noindent{}
%link
\begin{verbatim}
sage: Sha_conj = (L/24) * T^2 / (Om * Reg * c2 * c117223)
sage: Sha_conj
1.0
\end{verbatim}%link

Again we emphasize that we do not even know that 
the conjectural order computed above is a rational number. 

It seems almost a miracle that $L^{(4)}(E,1) = 214.65\ldots$,
$\Omega_E = 2.97\ldots$, and $\Reg(E) = 1.50\ldots$ have
anything to do with each other, but indeed they do:
\noindent{}
%link
\begin{verbatim}
sage: L/24, 2*Om*Reg
(8.9438473959, 8.9438473959)
\end{verbatim}
That these two numbers are the same to several decimal
places is a fact, independent of any conjectures.

\newpage
\section{The $p$-adic BSD Conjectural Formula}
Let $E$ be an elliptic curve over $\QQ$ and let $p$
be a prime of good ordinary reduction for $E$.

In Chapter~\ref{ch:rank} (see Theorem~\ref{thm:padiclseries}) 
we defined a $p$-adic $L$-series
$$
   \cL_p(E,T) \in \QQ_p[[T]].
$$
Conjecture~\ref{conj:mtt} asserted that
$
\ord_{T} \cL_p(E,T) = \rank E(\QQ).
$
Just as is the cases for $L(E,s)$, there is a conjectural
formula for the leading coefficient of the power series
$\cL_p(E,T)$.   This formula is due to Mazur, Tate,
and Teitelbaum \cite{mtt}.

First, suppose $\ord_T \cL_p(E,T) = 0$, i.e.,
$\cL_p(E,0) \neq 0$.  Recall that the interpolation property
\eqref{eq:interp} for $\cL_p(E,T)$ implies that
$$
  \cL_p(E,0) = \eps_p \cdot L(E,1)/\Omega_E,
$$
where 
\begin{equation}\label{eq:epsp}
\eps_p = (1-\alpha^{-1})^2,
\end{equation}
and $\alpha\in \ZZ_p$ is the unit root of $x^2 - a_p x + p=0$.
Thus the usual BSD conjecture predicts that if
the rank is $1$, then
\begin{equation}\label{eq:pbsd0}
  \cL_p(E,0) = \eps_p \cdot \frac{\prod_\ell c_\ell \cdot \#\Sha(E/\Q) \cdot \Reg(E)}{\#E(\Q)_{\tor}^2}
\end{equation}

Notice in \eqref{eq:pbsd0} that since $E(\QQ)$ has rank $0$, we have
$\Reg(E) = 1$, so there is no issue with the left hand side
being a $p$-adic number and the right hand side not making sense.
It would be natural to try to generalize \eqref{eq:pbsd0} to higher order
of vanishing as follows.  Let $\cL_p^*(E,0)$ denote the leading
coefficient of the power series $\cL_p(E,T)$.  Then 
\begin{equation}
  \cL_p^*(E,0) ``=\text{''} \eps_p \cdot \frac{\prod_\ell c_\ell \cdot \#\Sha(E/\Q) \cdot \Reg(E)}{\#E(\Q)_{\tor}^2}\qquad\text{(nonsense!!).}
\end{equation}
Unfortunately \eqref{eq:pbsd0} is total nonsense when the rank
is bigger than $0$.  The problem is that $\Reg(E)\in\R$ is a real number,
whereas $\eps_p$ and $\cL_p^*(E,0)$ are both $p$-adic numbers. 

The key {\em new idea} needed to make 
a conjecture is to replace the real-number regulator
$\Reg(E)$ with a $p$-adic regulator $\Reg_p(E) \in \Q_p$. 
This new regulator is defined in a way analogous to the classical
regulator, but where many classical complex analytic objects
are replaced by $p$-adic analogues.  Moreover, the $p$-adic regulator
was, until recently (see \cite{mazur-tate-stein}), much 
more difficult to compute than the classical real regulator. 
We will define the $p$-adic number $\Reg_p(E) \in \Q_p$
in the next section.  

\begin{conjecture}[Mazur, Tate, and Teitelbaum]\label{conj:mttformula}

  Let $E$ be an elliptic curve over $\QQ$ and let $p$ be a prime of
  good ordinary reduction for $E$.  Then the rank of $E$ equals 
  $\ord_T(\cL_p(E,T))$ and 
\begin{equation}
  \cL_p^*(E,0) = \eps_p \cdot \frac{\prod_\ell c_\ell \cdot \#\Sha(E/\Q) \cdot \Reg_p(E)}{\#E(\Q)_{\tor}^2},
\end{equation}
where $\eps_p$ is as in \eqref{eq:epsp}, and the $p$-adic regulator
$\Reg_p(E)\in\Q_p$ will be defined below.
\end{conjecture}
\begin{remark}
  There are analogous conjectures in many other cases, e.g., good
  supersingular, bad multiplicative, etc.  See \cite{shark} for more details.
\end{remark}

\subsection{Example: A Curve of Rank $2$}
We only consider primes $p$ of good ordinary reduction for a given
curve $E$ in this section.  If $E$ is an elliptic curve with analytic
rank $0$, then the $p$-adic and classical BSD conjecture are the same,
so there is nothing new to illustrate.  We will thus consider only
curves of rank $\geq 1$ in this section.

We consider the elliptic curve 446d1 of rank $2$ at the prime
$p=5$. 
\begin{verbatim}
sage: E = EllipticCurve('446d1'); p = 5; E
Elliptic Curve defined by y^2 + x*y  = x^3 - x^2 - 4*x + 4 
over Rational Field
\end{verbatim}%link

\noindent{}Next we verify that the rank is $2$, that $p$
is a good ordinary prime, and that there are $10$ points
on $E$ modulo $p$ (so $E$ is {\em ananomolous at $p$}, i.e.,
$p\mid \#E(\FF_p)$).
%link
\begin{verbatim}
sage: E.rank()
2
sage: E.is_ordinary(p)
True
sage: E.Np(p)
10
\end{verbatim}%link

\noindent{}Next we compute the $p$-adic $L$-series of $E$ at $p$.
We add $O(T^7)$ so that the displayed series doesn't take several lines.
% note -- make the precision bigger than 4 at some point!
%link
\begin{verbatim}
sage: Lp  = E.padic_lseries(p)
sage: LpT = Lp.series(4)
sage: LpT = LpT.add_bigoh(7); LpT
(5 + 5^2 + O(5^3))*T^2 + (2*5 + 3*5^2 + O(5^3))*T^3 
          + (4*5^2 + O(5^3))*T^4 + (4*5 + O(5^2))*T^5 
          + (1 + 2*5 + O(5^3))*T^6 + O(T^7)
\end{verbatim}%link

\noindent{}We compute the $p$-adic modular form $E_2$ evaluated
on our elliptic curve with differential $\omega$ to precision $O(p^{8})$.
This is the key difficult input to the computation of the $p$-adic
regulator $\Reg_p(E)$.
%link
\begin{verbatim}
sage: E.padic_E2(p, prec=8)
3*5 + 4*5^2 + 5^3 + 5^4 + 5^5 + 2*5^6 + 4*5^7 + O(5^8)
\end{verbatim}%link

\noindent{}We compute the normalized $p$-adic regulator,
normalized to the choice of $1+p$ as a topological
generator of $1+p\,\mathbb{Z}_p$.
% note -- this normalization will be automatic at some point soon.
%link
\begin{verbatim}
sage: Regp = E.padic_regulator(p, 10)
sage: R = Regp.parent()
sage: kg = log(R(1+p))
sage: reg = Regp * p^2 / log(R(1+p))^2
sage: reg*kg^2
2*5 + 2*5^2 + 5^4 + 4*5^5 + 2*5^7 + O(5^8)
\end{verbatim}%link

\noindent{}We compute the Tamagawa numbers and torsion subgroup.
%link
\begin{verbatim}
sage: E.tamagawa_numbers()
[2, 1]
sage: E.torsion_order()
1
\end{verbatim}%link

\noindent{}We compute $\cL_p^*(E,0)$, which is the leading term of 
the $p$-adic $L$-function.  It is not a 
unit, so we call the prime $p$ an {\em irregular} prime.
%link
\begin{verbatim}
sage: Lpstar = LpT[2]; Lpstar
5 + 5^2 + O(5^3)
\end{verbatim}%link

\noindent{}Finally, putting everything together
we compute the conjectural $p$-adic order of $\#\Sha(E/\QQ)$.
In particular, we see that conjecturally 
$\#\Sha(E/\QQ)(5)$ is trivial.
%link
\begin{verbatim}
sage: eps = (1-1/Lp.alpha(20))^2
sage: Lpstar / (eps*reg*(2*1)) * (1)^2
1 + O(5^2)
\end{verbatim}

\subsection{The $p$-adic Regulator}
Fix an elliptic curve $E$ defined over $\QQ$ and a prime $p$ of good
ordinary reduction for $E$.  In this section we define the $p$-adic
regulator $\Reg_p(E)$.  See \cite{mtt}, \cite{mazur-tate-stein} and
\cite{shark} and the references listed there for a more general
discussion of $p$-adic heights, especially for bad or supersingular
primes, and for elliptic curves over number fields.  See also
forthcoming work of David Harvey for highly optimized computation
of $p$-adic regulators. 

The $p$-adic logarithm $\log_p:\QQ_p^* \to (\QQ_p,+)$ is
the unique group homomorphism with $\log_p(p)=0$ that extends the
homomorphism $\log_p:1+p\ZZ_p \to \QQ_p$ defined by the usual power
series of $\log(x)$ about $1$.  Explicitly, if $x\in\QQ_p^*$, then
$$\log_p(x) = \frac{1}{p-1}\cdot \log_p(u^{p-1}),$$
where $u = p^{-\ord_p(x)} \cdot x$ is the unit part of~$x$, and the
usual series for $\log$ converges at $u^{p-1}$.

\begin{example}
For example, in \sage we compute the logs of a couple of
non-unit elements of $\QQ_5$ as follows:
\begin{verbatim}
sage: K = Qp(5,8); K
5-adic Field with capped relative precision 8
sage: a = K(-5^2*17); a
3*5^2 + 5^3 + 4*5^4 + 4*5^5 + 4*5^6 + 4*5^7 + 4*5^8 + 4*5^9 + O(5^10)
sage: u = a.unit_part()
3 + 5 + 4*5^2 + 4*5^3 + 4*5^4 + 4*5^5 + 4*5^6 + 4*5^7 + O(5^8)
sage: b = K(1235/5); b
2 + 4*5 + 4*5^2 + 5^3 + O(5^8)
sage: log(a)
5 + 3*5^2 + 3*5^3 + 4*5^4 + 4*5^5 + 5^6 + O(5^8)
sage: log(a*b) - log(a) - log(b)
O(5^8)
\end{verbatim}%link

\noindent{}Note that we can recover $b$:
%link
\begin{verbatim}
sage: c = a^b; c
2*5^494 + 4*5^496 + 2*5^497 + 5^499 + 3*5^500 + 5^501 + O(5^502)
sage: log(c)/log(a)
2 + 4*5 + 4*5^2 + 5^3 + O(5^7)
\end{verbatim}
\end{example}


Let $\cE$ denote the N\'eron model of~$E$ over~$\Z$.  Let $P\in E(\Q)$
be a non-torsion point that reduces to $0\in E(\F_p)$ and to the
connected component of $\cE_{\F_\ell}$ at all primes $\ell$ of bad
reduction for~$E$.  For example, given any point $Q\in E(\Q)$
one can construct such a $P$ by multiplying it by the least common
multiple of the Tamagawa numbers of $E$.

\begin{exercise}
Show that any
nonzero point $P=(x(P),y(P)) \in E(\Q)$ can be written uniquely in the
form $(a/d^2, b/d^3)$, where $a,b,d \in \Z$, $\gcd(a,d)=\gcd(b,d)=1$,
and $d>0$.
(Hint: Use that $\Z$ is a unique factorization domain.)
\end{exercise}
The function $d(P)$ assigns to $P$ this square root~$d$ of
the denominator of the $x$-coordinate $x(P)$.

\begin{example}
We compute a point on a curve, and observe that the denominator
of the $x$ coordinate is a perfect square. 
\begin{verbatim}
sage: E = EllipticCurve('446d1')
sage: P = 3*E.gen(0); P
(32/49 : -510/343 : 1)
\end{verbatim}
\end{example}

Let
\begin{equation}\label{eqn:xt}
  x(t) = \frac{1}{t^2} + \cdots \in \Z_p((t))
\end{equation}
be the formal power series that expresses $x$ in terms of the local
parameter $t=-x/y$ at infinity.  Similarly, let $y(t) = -x(t)/t$
be the corresponding series for $y$.
If we do the change of variables $t=-x/y$ and $w=-1/y$,
so $x=t/w$ and $y=-1/w$, then the Weierstrass equation
for $E$ becomes
$$
s  =  {t}^{3}  + {{a_{1}  s}  t} +
{{a_{2}  w}  {t}^{2} } +
{a_{3}  {w}^{2} }
+ {{a_{4}  {w}^{2} }  t}
+ {a_{6}  {w}^{3} } = F(w,t).
$$
Repeatedly substituting this equation into itself
recursively yields a power series expansion for
$w = -1/y$ in terms of $t$, hence for both $x$ and $y$.

\begin{remark}
The \defn{formal group} of $E$ is a power
series
$$F(t_1, t_2) \in R = \ZZ[a_1,\ldots, a_6][[t_1, t_2]].$$
defined as follows.
Since $x(t)$ and $y(t)$ satisfy the equation of $E$,
the 
points $P_1 = (x(t_1),y(t_1))$ and
$P_2 = (x(t_2),y(t_2))$
are in $E(R)$.  As explained explicitly in
\cite[\S{}IV.1]{silverman:aec}, their sum is
$$Q = P_1 + P_2 = (x(F), y(F)) \in E(R)$$
for some $F = F(t_1, t_2) \in R$.
\end{remark}
 

\begin{example}
We compute the above change of variables in SAGE:
\begin{verbatim}
sage: var('a1 a2 a3 a4 a6')
sage: E = EllipticCurve([a1,a2,a3,a4,a6]); E
Elliptic Curve defined by
     y^2 + a1*x*y + a3*y = x^3 + a2*x^2 + a4*x + a6
over Symbolic Ring
sage: eqn = SR(E); eqn
(y^2 + a1*x*y + a3*y) == (x^3 + a2*x^2 + a4*x + a6)
sage: F = eqn.lhs() - eqn.rhs(); F
y^2 + a1*x*y + a3*y - x^3 - a2*x^2 - a4*x - a6
sage: G = w^3*F(x=t/s, y=-1/w); G.expand()
-t^3 - a2*w*t^2 - a4*w^2*t - a1*w*t - a6*w^3 - a3*w^2 + w
\end{verbatim}
\end{example}

\begin{example}
We use \sage to compute the formal power series $x(t)$ and $y(t)$
for the rank $1$ elliptic curve 37a.
\begin{verbatim}
sage: E = EllipticCurve('37a'); E
Elliptic Curve defined by y^2 + y = x^3 - x over Rational Field
sage: F = E.formal_group(); F
Formal Group associated to the Elliptic Curve defined by
y^2 + y = x^3 - x over Rational Field
sage: x = F.x(prec=8); x
t^-2 - t + t^2 - t^4 + 2*t^5 - t^6 - 2*t^7 + O(t^8)
sage: y = F.y(prec=8); y
-t^-3 + 1 - t + t^3 - 2*t^4 + t^5 + 2*t^6 - 6*t^7 + O(t^8)
\end{verbatim}%link

\noindent{}Notice that the power series satisfy the equation of the curve.
%link
\begin{verbatim}
sage: y^2 + y == x^3 - x
True
\end{verbatim}
%link
\end{example}

Recall that $\omega_E = \frac{dx}{2y + \ua_1 x + \ua_3}$ is the
differential on a fixed choice of Weierstrass equation for
$E$. Let
$$
 \omega(t) = \frac{dx}{2y + \ua_1 x + \ua_3} \in \QQ((t)) dt
$$
be the formal invariant holomorphic differential on $E$.
\begin{example}
Continuing the above example, we compute the formal
differential on $E$:
%link
\begin{verbatim}
sage: F.differential(prec=8)
1 + 2*t^3 - 2*t^4 + 6*t^6 - 12*t^7 + O(t^8)
\end{verbatim}%link

\noindent{}We can also compute $\omega(t)$ directly from the definition:
%link
\begin{verbatim}
sage: x.derivative()/(2*y+1)
1 + 2*t^3 - 2*t^4 + 6*t^6 - 12*t^7 + 6*t^8 + 20*t^9 + O(t^10)
\end{verbatim}
\end{example}

The following theorem, which is proved in \cite{mazur-tate:sigma},
uniquely determines a power series~$\sigma\in t\ZZ_p[[t]]$ and constant
$c\in\ZZ_p$.
\begin{theorem}[Mazur-Tate]\label{thm:uniqde}
  There is exactly one odd function $\sigma(t) = t + \cdots \in
  t\Z_p[[t]]$ and constant $c\in \Z_p$ that together satisfy the
  differential equation
\begin{equation}\label{eqn:sigmadef}
x(t)
+ c = -\frac{d}{\omega}\left( \frac{1}{\sigma}
  \frac{d\sigma}{\omega}\right),
\end{equation}
where $\omega$ is the invariant differential
$dx/(2y+a_1x+a_3)$ associated with our chosen Weierstrass equation
for $E$.
\end{theorem}

The above theorem produces a (very inefficient) algorithm to compute
$c$ and $\sigma(t)$.  Just view $c$ as a formal indeterminate and
compute $\sigma(t) \in \Q[c][[t]]$, then obtain constraints on $c$
using that the coefficients of $\sigma$ must be in $\Z_p$.  These
determine $c$ to some precision, which increases as we compute
$\sigma(t)$ to higher precision.  Until recently this was the only
known way to compute $c$ and $\sigma(t)$ -- fortunately the method
of \cite{mazur-tate-stein} is much faster in general.

\begin{definition}[Canonical $p$-adic Height]
Let $E$ be an elliptic curve over $\QQ$ with good ordinary reduction
at the odd prime $p$.  Let $\log_p$, $d$, and $\sigma(t)$ be
as above and suppose $P \in E(\QQ)$ and that $nP$ is a nonzero multiple
of $P$ such that $nP$ reduces to the identity component of the N\'eron
model of $E$ at each prime of bad reduction.  Then the
\defn{$p$-adic canonical height} of $P$ is 
$$
h_p(P) = \frac{1}{n^2}\cdot \frac{1}{p} \cdot \log_p\left(\frac{\sigma(P)}{d(P)}\right).
$$
\end{definition}

\begin{definition}[$p$-adic Regulator]
The \defn{$p$-adic regulator} of $E$ is the discriminant (well defined up to sign)
of the bilinear $\QQ_p$-valued pairing
$$
  (P,Q)_p = h_p(P) + h_p(Q) - h_p(P+Q).
$$
\end{definition}

\begin{conjecture}[Schneider]
The $p$-adic regulator $\Reg_p(E)$ is nonzero.
\end{conjecture}

\begin{theorem}[Kato, Schneider, et al.]
Let $E$ be an elliptic curve over $\QQ$ with good ordinary reduction
at the odd prime $p$ and assume that the $p$-adic
Galois representation $\rho_{E,p}$ is surjective.
If
$$\ord_T(\cL_p(E,T)) \leq  \rank E(\Q),$$
then $\#\Sha(E/\Q)(p)$ is finite.
Moreover, if $\Reg_p(E)$ is nonzero, then
$$
\ord_p(\#\Sha(E/\Q)(p)) \leq \ord_p\left(\frac{\cL_p^*(E,0)}{\prod c_\ell \cdot \Reg_p(E)}\right).
$$
\end{theorem}




% \newpage
% \section{The $p$-adic Regulator}
% Let $E$ be an elliptic curve over $\QQ$ and let $p$ be an odd prime of
% good ordinary reduction for $E$.  We define the $p$-adic regulator
% using a $p$-adic analogue of the height pairing.
 
% The $p$-adic de Rham cohomology $D_p(E) = \H^1_{\dR}(E/\Q)\tensor\Q_p$ of $E$
% has basis $\omega_E, \eta_E = x\omega_E$, where $\omega_E$ is the 
% minimal differential on a fixed choice of Weierstrass equation for $E$.
% Let $\nu$ be an element of $D_p(E)$.
% We will define a $p$-adic height function 
% $h_\nu\colon E(\QQ)\rTo \QQ_p$ which depends linearly on the
% vector $\nu$. Hence it is sufficient to define it on the basis
% $\omega=\omega_E$ and $\eta=\eta_E$.
 
% If $\nu=\omega$, then we define 
%  \begin{equation*}
%   h_\omega(P)=-\log_E(P)^2,
%  \end{equation*}
% where $\log_E$ is the linear extension of the $p$-adic elliptic 
% logarithm $\log_{\hat E}\colon \hat E(p\ZZ_p)\rTo p\ZZ_p$ defined 
% on the formal group $\hat E$. 
 
%  For $\nu=\eta$, we define first the $p$-adic sigma function of Bernardi $\sigma(z)$ as in~\cite{bernardi}. Denote by $t=-\tfrac{x}{y}$ the uniformizer at $\ZeroE$. %and write $z(t) = \log_{\hat E}(t)$. 
% Define the Weierstrass $\wp$-function as usual by 
%  \begin{equation*}
%   \wp(t) = x(t)+\frac{a_1^2+4\,a_2}{12} \in\QQ((t))\,.
%  \end{equation*}

% Here $a_1$ and $a_2$ are the coefficients of the minimal Weierstrass
% equation~\eqref{w_eq} of $E$. The function $\wp(t)$ is a solution to
% the usual differential equation. We define the sigma-function of
% Bernardi to be a solution of the equation\william{``the'' means it is unique; ``a'' means
% it isn't unique -- which is it?  Also, is there any integrality condition?  I think it's
% not unique, so we should say ``a'', and say what it is unique up to (a constant?). Christian: Now it is unique.}
%  \begin{equation*}
%    - \wp(t)  = \frac{d}{\omegaE}\left(\frac{1}{\sigma}\cdot\frac{d\sigma}{\omegaE}\right)
%  \end{equation*}
%  such that $\sigma(0)=0$, $\frac{d\sigma}{dt}(0) =1$, and $\sigma(t(-P))=-\sigma(t(P))$.
%  This provides us with a series 
%  \begin{equation*}
%   \sigma(t) = t + \frac{a_1}{2}\,t^2 + \frac{a_1^2+a_2}{3}\,t^3+\frac{a_1^3+2a_1a_2+3a_3}{4}\,t^4+\cdots  \in \QQ(\!(t)\!)\,.
%  \end{equation*}
%  As a function on the formal group $\hat E(p\ZZ_p)$ it converges for $\ord_p(t) > \tfrac{1}{p-1}$.
 
%   Given a point $P$ in $E(\QQ)$ there exists a multiple $m\cdot P$ such that $\sigma(t(P))$ converges and such that $m\cdot P$ has good reduction at all primes. Denote by $e(m\cdot P)\in\ZZ$ the square root of the denominator of the $x$-coordinate of $m\cdot P$.  Now define
%   \begin{equation*}
%    h_{\eta}(P) = \frac{2}{m^2} \cdot \log_p\left (\frac{\sigma(t(m\cdot P))}{e(m\cdot P)}\right )\,.
%   \end{equation*}
%  It is proved in~\cite{bernardi} that this function is quadratic and satisfies the parallelogram law.
 
%   Finally, if $\nu= a\, \omega+b\,\eta$ then put
%   \begin{equation*}
%    h_\nu(P) = a \, h_{\omega}(P) + b\, h_{\eta}(P)\,.
%   \end{equation*}
%   This quadratic function induces a bilinear symmetric pairing $\langle\cdot,\cdot\rangle_{\nu}$ with values in $\QQ_p$.
  
%   \subsection{The good ordinary case}
%   Since we have only a single $p$-adic $L$-function in the case that the reduction is good ordinary, we have also to pin down a canonical choice of a $p$-adic height function. This was first done by Schneider~\cite{schneider1} and Perrin-Riou~\cite{pr82}. We refer to~\cite{mt} and~\cite{mst} for more details.
  
%   Let $\nu_{\alpha}= a \, \omega + b\,\eta$ be an eigenvector of $\varphi$ on $D_p(E)$ associated to the eigenvalue $\tfrac{1}{\alpha}$. The value $e_2 =\mathbf{E}_2(E,\omegaE) = -12\cdot \tfrac{a}{b}$ is the value of the Katz $p$-adic Eisenstein series of weight $2$ at $(E,\omegaE)$.
%   Then, if $P$ has good reduction at all primes and lies in the range of convergence of $\sigma(t)$, we define the canonical $p$-adic height of $P$ to be
%   \begin{align}
%    \hat h_p (P) &= \frac{1}{b}\cdot h_{\nu_{\alpha}}(P) \notag\\ 
%    				&= -\frac{a}{b} \cdot \logE(P)^2 +2\, \log\left (\frac{\sigma(t(P))}{e( P)}\right ) \notag\\
% 				&= 2\,\log_p \left ( \frac{\exp(\frac{e_2}{24} \logE(P)^2)\cdot \sigma(t(P))}{e(P)} \right) = 2\, \log_p \left ( \frac{\sigma_p(t(P))}{e(P)} \right)\,. \label{hp_eq}
% \end{align} 

% The function $\sigma_p(t)$, defined by the last line, is called the
% canonical sigma-function, see~\cite{mt}, it is known to lie in
% $\ZZ_p[\![t]\!]$.  The $p$-adic height defined here is up to the
% factor of $2$ the same as in~\cite{mst}.\footnote{This factor is
%   needed if one does not want to modify the $p$-adic version of the
%   Birch and Swinnerton-Dyer conjecture~\ref{pbsd_ord_con}.}
  
% We write $\langle \cdot,\cdot\rangle_p$ for the canonical $p$-adic
% height pairing on $E(\QQ)$ associated to $\hat h_p$, and we
% write $\Reg_p(E/\QQ)$ for the discriminant of the height pairing 
% on $E(\QQ)/E(\QQ)_{\tor}$, which is well defined up to sign. 
  
% \begin{conjecture}{Schneider~\cite{schneider1}}\label{conreg_con}
%   The canonical $p$-adic height is non-de\-ge\-ne\-ra\-te on 
%   $E(\QQ)/E(\QQ)_{\tor}$. In other words, the canonical $p$-adic regulator
%   $\Reg_p(E/\QQ)$ is nonzero.
% \end{conjecture}
  
% Apart from the special case treated in~\cite{bertrand} of curves with
% complex multiplication of rank $1$, there are hardly any results on
% this conjecture. See also~\cite{wuth04}.


%%% Local Variables: 
%%% mode: latex
%%% TeX-master: "main"
%%% End: 
