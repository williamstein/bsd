\chapter{The BSD Rank Conjecture}\label{ch:rank}
This chapter explains the conjecture that Birch and Swinnerton-Dyer
made about ranks of elliptic curves (the BSD rank conjecture).

\section{Statement of the BSD Rank Conjecture}\label{sec:bsdrank}

An excellent reference for this section is Andrew Wiles's 
Clay Math Institute paper \cite{wiles:cmi}.   The reader
is also strongly encouraged to look Birch's original
paper \cite{birch:bsd} to get a better sense of the excitement
surrounding this conjecture, as exemplified in the following
quote:
\begin{quote}
  ``I want to describe some computations undertaken by myself and
  Swinnerton-Dyer on EDSAC by which we have calculated the
  zeta-functions of certain elliptic curves. As a result of these
  computations we have found an analogue for an elliptic curve of the
  Tamagawa number of an algebraic group; and conjectures (due to
  ourselves, due to Tate, and due to others) have proliferated.''
\end{quote}

An \defn{elliptic curve} $E$ over a field $K$
is the projective closure
of the zero locus of a nonsingular affine curve 
\begin{equation}\label{eqn:weq}
   y^2 + a_1 xy + a_3 y = x^3 + a_2 x^2 + a_4 x + a_6,
\end{equation}
where $a_1,a_2,a_3,a_4,a_6\in K$.
There is a simple algebraic condition on the $a_i$ that 
ensures that \eqref{eqn:weq} defines a nonsingular curve
(see, e.g., \cite{silverman:aec}).

An elliptic curve $E$ has genus $1$, and the set of points on $E$ has
a natural structure of {\em abelian group}, with identity element the one
extra projective point at~$\infty$.  Again, there are simple algebraic
formulas that, given two points $P$ and $Q$ on an elliptic curve,
produce a third point  $P+Q$  on the elliptic curve.  Moreover,
if $P$ and $Q$ both have coordinates in $K$, then so does $P+Q$.
The \defn{Mordell-Weil group}
$$
  E(K) = \{\text{ points on $E$ with coordinates in $K$ }\}
$$
of $E$ over $K$ plays a central role in this book. 

In the 1920s, Mordell proved that if $K=\Q$, then $E(\Q)$ is finitely
generated, and soon after Weil proved that $E(K)$ is finitely
generated for any number field $K$, so
\begin{equation}\label{eqn:rt}
   E(K) \ncisom \Z^r \oplus T,
 \end{equation}
 where $T$ is a finite group.
Perhaps the chief invariant of an elliptic curve $E$ over a number
field $K$ is the \defn{rank}, which is the number $r$ in
\eqref{eqn:rt}.

Fix an elliptic curve $E$ over $\Q$.  For all but finitely
many prime numbers $p$, the equation \eqref{eqn:weq} 
reduces modulo $p$ to  define an elliptic curve 
over the finite field $\F_p$.  
The primes that must be excluded are exactly the primes that divide
the discriminant $\Delta$ of \eqref{eqn:weq}.

As above, the set
of points $E(\F_p)$ is an abelian group.  This group
is finite, because it is contained in the set $\P^2(\F_p)$ of
rational points in the projective plane.    Moreover,
since it is the set of points on a (genus 1) curve,
a theorem of Hasse implies that
$$
    |p+1 - \#E(\F_p) | \leq 2\sqrt{p}.
$$
The error terms
$$
   a_p = p+1 - \#E(\F_p)
$$
play a central role in almost everything in this book.
We next gather together the error terms into a single
``generating function'':
$$
  \tilde{L}(E,s) = \prod_{p\nmid \Delta}
     \left( \frac{1}{1 - a_p p^{-s} + p^{1-2s}}\right).
$$
The function $\tilde{L}(E,s)$ defines a complex
analytic function on some right half plane $\Re(s)>\frac{3}{2}$.

A deep theorem of Wiles et al. \cite{wiles:fermat,
  breuil-conrad-diamond-taylor}, which many consider the crowning
achievement of 1990s number theory, implies that
$\tilde{L}(E,s)$ can be analytically continued to an analytic function
on all~$\C$.  This implies that $\tilde{L}(E,s)$ has a Taylor series
expansion about $s=1$:
$$
    \tilde{L}(E,s) = c_0  + c_1 (s-1) + c_2 (s-1)^2 + \cdots 
$$
Define the \defn{analytic rank} $r_{\an}$ of $E$ to be the order of vanishing
of $\tilde{L}(E,s)$ as $s=1$, so 
$$
    \tilde{L}(E,s) =  c_{r_{\an}} (s-1)^{r_{\an}} + \cdots.
$$
The definitions of the analytic and Mordell-Weil ranks
could not be more different -- one is completely analytic and the
other is purely algebraic. 

\begin{conjecture}[Birch and Swinnerton-Dyer Rank Conjecture]\label{conj:bsdrank}
Let $E$ be an elliptic curve over $\Q$.  Then the
algebraic and analytic ranks of $E$ are the same. 
\end{conjecture}

{\em This problem is extremely difficult.}  The conjecture was made
in the 1960s, and hundreds of people have thought about it
for over 4 decades.  
The work of Wiles et al. on modularity in late 1999,
combined with earlier work of Gross, Zagier, and Kolyvagin,
and many others proves the following partial result toward
the conjecture.

\begin{theorem}
Suppose $E$ is an elliptic curve over $\Q$ and that $r_{\an} \leq 1$. 
Then the algebraic and analytic ranks of $E$ are the same. 
\end{theorem}

In 2000, Conjecture~\ref{conj:bsdrank} was declared a million dollar
millenium prize problem by the Clay Mathematics Institute, which
motivated even more work, conferences, etc., on the conjecture.  Since
then, to the best of my knowledge, not a single new result directly
about Conjecture~\ref{conj:bsdrank} has been 
proved\footnote{Much interesting new work
has been done on related conjectures and problems.}.  
The class of curves for which
we know the conjecture is still the set of curves over $\Q$ with
$r_{\an} \leq 1$, along with a finite set of individual curves on
which further computer calculations have been performed (by Cremona,
Watkins, myself, and others).

\begin{quote}
\noindent{\em ``A new idea is needed.''} 

-- Nick Katz on BSD, at a 2001 Arizona Winter School
\end{quote}

And another quote from Bertolini-Darmon (2001):
\begin{quote}
``The following question stands as the ultimate
challenge concerning the Birch and Swinnerton-Dyer conjecture
for elliptic curves over $\QQ$: {\em Provide evidence for
the Birch and Swinnerton-Dyer conjecture in cases where~$\ord_{s=1} L(E,s)~>~1$.}''
\end{quote}


\section{The BSD Rank Conjecture Implies that $E(\Q)$ is Computable}\label{sec:bsdimpliescomputable}


 \begin{proposition}\label{prop:bsdalgrank}
Let $E$ be an elliptic curve over $\QQ$. 
 If Conjecture~\ref{conj:bsdrank} is true, then 
there is an algorithm to compute the rank of $E$.
 \end{proposition}
 \begin{proof}
   By naively searching for points in $E(\QQ)$ we obtain a lower bound
   on $r$, which is closer and closer to the true rank $r$, the longer
   we run the search.  At some point this lower bound will equal $r$,
   but without using further information we do not know when that will
   occur.  

   As explained, e.g., in \cite{cremona:algs} (see also
   \cite{dokchitser:lfun}), we can for any $k$ compute
   $L^{(k)}(E,1)$ to any desired precision.  Such computations yield
   upper bounds on $r_{\an}$.  In particular, if we compute
   $L^{(k)}(E,1)$ and it is nonzero (to the precision of our
   computation), then $r_{\an} \leq k$.  Eventually this method will
   also converge to give an upper bound on $r_{\an}$, though again
   without further information we do not know when our computed upper
   bound on $r_{\an}$ equals to the true value of $r_{\an}$.  

   Since we are assuming that Conjecture~\ref{conj:bsdrank} is true, we know
   that $r = r_{\an}$, hence at some point the lower bound on $r$
   computed using point searches will equal the upper bound on
   $r_{\an}$ computed using the $L$-series.  At this point, by
   Conjecture~\ref{conj:bsdrank}, we know the true value of $r$.  
   \end{proof}

Next we show that given the rank $r$, the
full group $E(\QQ)$ is computable.  The issue is that
what we did above might have only computed a subgroup
of finite index.  The argument below follows
\cite[\S3.5]{cremona:algs} closely.

The \defn{naive height} $h(P)$ of a point $P =(x,y) \in E(\QQ)$
is $$h(P) = \log(\max(\numer(x),\denom(x))).$$ The \defn{N\'eron-Tate 
canonical height} of $P$ is 
$$
  \hat{h}(P) = \lim_{n\to\infty} \frac{h(2^n P)}{4^n}.
$$
Note that if $P$ has finite order then $\hat{h}(P)=0$.
Also, a standard result  is 
 that the \defn{height pairing}
$$
\langle P,Q\rangle =  \frac{1}{2}\left( \hat{h}(P+Q) - \hat{h}(P) - \hat{h}(Q)\right)
$$
defines a nondegenerate real-valued quadratic form on $E(\QQ)/_{\tor}$
with discrete image.

\begin{lemma}\label{lem:allgen}
Let $B>0$ be a positive real number such that 
$$
   S = \{ P \in E(\QQ)\, : \, \hat{h}(P) \leq B \}
$$
contains a set of generators for $E(\QQ)/2 E(\QQ)$.  Then
$S$ generates $E(\QQ)$.
\end{lemma}
\begin{proof}
  Let $A$ be the subgroup of $E(\QQ)/_{\tor}$ generated by the points
  in $S$.  Suppose for the sake of contradiction that $A$ is a proper
  subgroup. Then there is $Q\in E(\Q)\setminus A$ with $\hat{h}(Q)$
minimal, since $\hat{h}$ takes a discrete set of values.
Since $S$ contains generators for $E(\QQ)/2 E(\QQ)$, there is
an element $P \in S$ that is congruent to $Q$ modulo $2 E(\QQ)$, 
i.e., so that
   $$
      Q = P + 2R,
$$
for some $R\in E(\QQ)$.
We have $R\not\in A$ (since otherwise $Q$ would be in $A$), so
$\hat{h}(R)\geq \hat{h}(Q)$ by minimality.  
Finally, since $\hat{h}$ is quadratic and nonnegative, we have
\begin{align*}
 \hat{h}(P) &= \frac{1}{2}\left(\hat{h}(Q+P) + \hat{h}(Q-P) - \hat{h}(Q)\right)\\
   &\geq \frac{1}{2} \hat{h}(2R) - \hat{h}(Q)\\
  &=2\hat{h}(R) - \hat{h}(Q) \geq \hat{h}(Q) > B.
\end{align*}
(Here we use that $\hat{h}(P) =\langle P, P\rangle$
and use properties of a bilinear form.)
\end{proof}

\begin{proposition}
Let $E$ be an elliptic curve over $\QQ$. 
 If Conjecture~\ref{conj:bsdrank} is true, then 
there is an algorithm to compute $E(\QQ)$.
\end{proposition}
\begin{proof}
  By Proposition~\ref{prop:bsdalgrank} we can compute the rank $r$ of
  $E(\QQ)$.  Note that we can also trivially compute the subgroup
  $E(\QQ)[2]$ of elements of order $2$ in $E(\QQ)$, since if $E$ is
  given by $y^2=x^3+ax+b$, then this subgroup is generated by points
  $(\alpha, \beta)$, where $\alpha$ is a rational root of $x^3+ax+b$.
  Thus we can compute $s = \dim_{\F_2} E(\QQ)/2 E(\QQ)$, since it is equal to $r
  + \dim E(\QQ)[2]$.  

  Run any search for points in $E(\QQ)$ and use that $\hat{h}$ is a
  nondegenerate quadratic form to find independent points $P_1,\ldots,
  P_r$ of infinite order.  It is easy to check whether a point $P$ is
  twice another point (just solve a relatively simple algebraic
  equation).  Run through all subsets of the points $P_i$, and if any
  subset of the $P_i$ sums to $2Q$ for some point $Q\in E(\QQ)$, then we
  replace one of the $P_i$ by $Q$ and decrease the index of
  our subgroup in $E(\QQ)$ by a factor of $2$.  Because $E(\QQ)$ is a
  finitely generated group, after a finite number of steps (and
  including the $2$-torsion points found above) we obtain independent
  points $P_1,\ldots, P_s$ that generate $E(\QQ)/2 E(\QQ)$.

  Let $C$ the the explicit bound of Cremona-Pricket-Siksek on the
  difference between the naive and canonical height (i.e., for any
   $P \in E(\Q)$, we have $|h(P) - \hat{h}(P)| < C$). 
Let
 $$
    B = \max\{\hat{h}(P_1), \ldots, \hat{h}(P_s)\}.
$$
Then by a point search up to naive height $B+C$, we compute a set that
contains the set $S$ in Lemma~\ref{lem:allgen}.  This set
then contains generators for $E(\Q)$, hence we have computed $E(\Q)$.

   \end{proof}

\section{The Complex $L$-series $L(E,s)$}\label{sec:complexles}
In Section~\ref{sec:bsdrank} we defined a function $\tilde{L}(E,s)$,
which encoded information about $E(\F_p)$ for all but finitely
many primes $p$.  In this section we define the function $L(E,s)$,
which includes information about all primes, and the function
$\Lambda(E,s)$ that also includes information ``at infinity''. 

Let $E$ be an elliptic curve over $\QQ$ defined by {\em a}
\defn{minimal Weierstrass equation}
\begin{equation}\label{eqn:minweq}
  y^2 + a_1 xy + a_3 y = x^3 + a_2 x^2 + a_4 x + a_6.
\end{equation}
A minimal Weierstrass equation in one for which the $a_i$
are all integers and the discriminant $\Delta\in\ZZ$ is minimal
amongs all discriminants of Weierstrass equations for $E$
(again, see \cite{silverman:aec} for the definition of the
discriminant of a Weierstrass equation, and also for an
explicit description of the allowed transformations of 
a Weierstrass equation). 

For each prime number $p\nmid \Delta$, the equation 
\eqref{eqn:minweq} reduces modulo~$p$ to define an elliptic 
$E_{\F_p}$ over the finite field $\F_p$.  Let
$$
   a_p = p + 1 - \#E(\F_p).
$$
For each prime $p\mid \Delta$, we use the following
recipe to define $a_p$.  If the {\em singular} curve
$E_{\F_p}$ has a cuspidal singularity, e.g., is
$y^2 = x^3$, then let $a_p = 0$.  If it has a a nodal
singularity, e.g., like $y^2=x^3+x^2$, let $a_p=1$
if the slope of the tangent line at the singular
point is in $\F_p$ and let $a_p=-1$ if the slope is
not in $\F_p$.  Summarizing:
$$
  a_p = \begin{cases} 
0 & \text{if the reduction is cuspidal (``additive'')},\\
1 & \text{if the reduction is nodal and tangent line is $\F_p$-rational (``split multiplicative'')}\\
-1 &\text{if the reduction is nodal and tangent line is not $\F_p$-rational (``non-split multiplicative'')}
\end{cases}
$$

Even in the cases when $p\mid \Delta$, we still have
$$
  a_p = p + 1 - \#E(\F_p).
$$
When $E$ has additive reduction, the nonsingular points 
form a group isomorphic to $(\F_p,+)$, and there is
one singular point, hence $p+1$ points,
so 
$$
  a_p = p+1 - (p+1))) = 0.
$$ When $E$
has split multiplicative reduction, there is 1 singular
point plus the number of elements of a group isomorphic to $(\F_p^*, \times)$,
so $1 + (p-1) = p$ points, and 
$$
  a_p = p+1-p = 1.
$$
When $E$ has non-split multiplicative
reduction, there is $1$ singular point plus the number
of elements of a group isomorphic $(\F_{p^2}^*/\F_p^*, \times)$,
i.e., $p+2$ points, and 
$$
  a_p = p+1 - (p+2) = -1.
$$

The definition of the full $L$-function of $E$ is then
$$
 L(E,s) = \prod_{p\mid \Delta} \frac{1}{1-a_p p^{-s}} \cdot
         \prod_{p\nmid \Delta} \frac{1}{1-a_p p^{-s} + p\cdot p^{-2s}}.
  = \sum_{n=1}^{\infty} \frac{a_n}{n^s}.
$$

If in addition we add in a few more analytic factors to the $L$-function
we obtain a function $\Lambda(E,s)$ that satisfies a remarkably simple
functional equation.  
Let 
$$
  \Gamma(z) = \int_{0}^{\infty} t^{z-1} e^{-t} dt
$$
be the \defn{$\Gamma$-function} (e.g., $\Gamma(n) = (n-1)!$),
which defines a meromorphic function on $\C$, with poles
at the non-positive integers. 
\begin{theorem}[Hecke, Wiles et al.]\label{thm:functional}
  There is a unique positive integer $N=N_E$ and sign $\eps=\eps_E\in\{\pm 1\}$
  such that the function
$$
  \Lambda(E,s) = N^{s/2}\cdot (2\pi)^{-s}\cdot \Gamma(s)\cdot L(E,s)
$$
extends to a complex analytic function on all $\C$ that
satisfies the functional equation
\begin{equation}\label{eqn:functional}
  \Lambda(E,2-s) = \eps \cdot \Lambda(E,s),
\end{equation}
for all $s\in \CC$.
\end{theorem}
\begin{proof}
Wiles et al. prove that $L(E,s)$ is the $L$-series
attached to a modular form (see Section~\ref{sec:modularity} below),
and Hecke proved that the $L$-series of a modular
form analytically continues and satisfies the given
functional equation.
\end{proof}

The integer $N=N_E$ is called the \defn{conductor} of
$E$ and $\eps=\eps_E$ is called the \defn{sign in the functional
equation} for $E$ or the \defn{root number} of $E$.
One can prove that the primes that divide $N$ are the same
as the primes that divide $\Delta$.  Moreover, for $p\geq 5$,
we have that 
$$
\ord_p(N) = \begin{cases}
   0,& \text{if $p\nmid \Delta$},\\
   1,&\text{if $E$ has multiplicative reduction at $p$, and}\\
   2,&\text{if $E$ has additive reduction at $p$}.
 \end{cases}
 $$
There is a geometric algorithm called Tate's algorithm that
computes $N$ in all cases and $\eps$.   

\begin{example}
Consider the elliptic curve $E$ defined by
$$
  	y^2 + y = x^3 + 50x + 31.
$$
The above Weierstrass equation is minimal and
has discriminant
$$
 -1 \cdot 5^{6} \cdot 7^{2} \cdot 11.
$$
\begin{verbatim}
sage: e = EllipticCurve('1925d'); e
Elliptic Curve defined by y^2 + y = x^3 + 50*x + 31 over Rational Field
sage: e.is_minimal()
True
sage: factor(e.discriminant())
-1 * 5^6 * 7^2 * 11
\end{verbatim}%link

\noindent{}At $5$ the curve has additive reduction so $a_5 = 0$.
At $7$ the curve has split multiplicative reduction
so $a_7 = 1$.  At $11$ the curve has nonsplit multiplicative
reduction, so $a_{11} = -1$.  Counting points for $p=2,3$,
we find that
$$
 L(E,s) =     	
\frac{1}{1^{-s}} + \frac{3}{3^{-s}} + \frac{-2}{4^{-s}} + \frac{1}{7^{-s}} +
\frac{6}{9^{-s}} + \frac{-1}{11^{-s}} + \frac{-6}{12^{-s}} + \cdots
$$
%link
\begin{verbatim}
sage: [e.ap(p) for p in primes(14)]
[0, 3, 0, 1, -1, 4]
\end{verbatim}
\end{example}

\begin{corollary}\label{cor:parity}
Let $E$ be an elliptic curve over $\QQ$, let $\eps \in \{1, -1\}$
be the sign in the functional equation \eqref{eqn:functional},
and let $r_{E,\an} = \ord_{s=1} L(E,s)$.
Then $$\eps = (-1)^{r_{E,\an}}.$$
\end{corollary}
\begin{proof}
Because $\Gamma(1)=1$, we have $\ord_{s=1} L(E,s) = \ord_{s=1} \Lambda(E,s)$.
It thus sufficies to prove the corollary with $L(E,s)$ 
replaced by $\Lambda(E,s)$.  Note that $r = r_{E,\an}$ is the minimal
integer $r\geq 0$ such that $\Lambda^{(r)}(E,1)\neq 0$.
By repeated differentiation, we see that for any integer $k\geq 0$,
we have
\begin{equation}\label{eqn:parity}
  (-1)^k \Lambda^{(k)}(E, 2-s) = \eps\cdot \Lambda^{(k)}(s).
\end{equation}
Setting $s=1$ and $k=r$, and using that 
$\Lambda^{(r)}(E,1)\neq 0$, shows that $(-1)^r = \eps$, as claimed. 
\end{proof}

\begin{conjecture}[The Parity Conjecture]\label{conj:parity}
  Let $E$ be an elliptic curve over $\QQ$, let $r_{E,\an}$ be the
  analytic rank and $r_{E,\alj}$ be the algebraic rank. Then
$$
  r_{E,\alj} \con r_{E,\an} \pmod{2}.
$$
\end{conjecture}
Jan Nekovar has done a huge amount of work toward
Conjecture~\ref{conj:parity}; in particular, he proves it under the
(as yet unproved) hypothesis that $\Sha(E)$ is finite (see
Section~\ref{sec:sha} below).


\section{Computing $L(E,s)$}
In this section we briefly describe one way to evaluate $L(E,s)$,
for $s$ real.   See \cite{dokchitser:lfun} for a more sophisticated
analysis of computing $L(E,s)$ and its Taylor expansion for any
complex number $s$.

\begin{theorem}[Lavrik]\label{thm:computeL}
We have the following
rapidly-converging series expression for $L(E,s)$, for any 
complex number $s$:
$$
L(E,s) = N^{-s/2}\cdot (2\pi)^s\cdot  \Gamma(s)^{-1}\cdot
\sum_{n=1}^{\infty} a_n \cdot \left(F_n(s-1) - \eps F_n(1-s)\right)
$$
where
$$
  F_n(t) = 
           \Gamma\left(t+1,\, \frac{2\pi n}{\sqrt{N}}\right)
\cdot \left(\frac{\sqrt{N}}{2\pi n}\right)^{t+1},
$$
and
$$
  \Gamma(z,\alpha) = \int_{\alpha}^{\infty} t^{z-1} e^{-t}dt
$$
is the \defn{incomplete $\Gamma$-function}.
\end{theorem}

Theorem~\ref{thm:computeL} above is a special case of a more
general theorem that gives rapidly converging series
that allow computation of any Dirichlet series $\sum a_n n^s$
that meromorphically continues to the whole complex plane and
satisfies an appropriate functional equation.  
For more details, see \cite[\S10.3]{cohen:advanced},
especially Exercise~24 on page 521 of \cite{cohen:advanced}.


\subsection{Approximating the Rank}
Fix an elliptic curve $E$ over~$\Q$.  The usual method to {\em
  approximate} the rank is to find a series that rapidly converges to
$L^{(r)}(E,1)$ for $r=0,1,2,3,\ldots$, then compute $L(E,1)$,
$L'(E,1)$, $L^{(2)}(E,1)$, etc., until one appears to be nonzero. 
Note that half of the $L^{(k)}(E,1)$ are automatically $0$
because of equation \eqref{eqn:parity}.
For more details, see \cite[\S2.13]{cremona:algs}
and \cite{dokchitser:lfun}. 

In this section, we describe a slightly different method, which only
uses Theorem~\ref{thm:computeL} and the definition of the derivative.

\begin{proposition}
Write 
$$
   L(E,s) = c_r(s-1)^r + c_{r+1}(s-1)^{r+1} + \cdots.
$$
with $c_r\neq 0$.  Then 
$$ 
  \lim_{s\ra 1}\,
   (s-1)\cdot \frac{L'(E,s)}{L(E,s)} = r.
$$
\end{proposition}
\begin{proof}
Setting $L(s) = L(E,s)$, we have
\begin{align*}
 \lim_{s\ra 1} \, 
  (s-1)\cdot \frac{L'(s)}{L(s)}
   &=  \lim_{s\ra 1} \,
         (s-1)\cdot \frac{r c_r (s-1)^{r-1} + (r+1) c_{r+1}(s-1)^r + \cdots}
          {c_r(s-1)^r + c_{r+1}(s-1)^{r+1} + \cdots}\\
   &=  r\cdot \lim_{s\ra 1} \,
          \frac{c_r (s-1)^{r} + \frac{(r+1)}{r} c_{r+1}(s-1)^{r+1} + \cdots}
          {c_r(s-1)^r + c_{r+1}(s-1)^{r+1} + \cdots}\\
   &= r.
\end{align*}
\end{proof}

Thus the rank~$r$ is 
the limit as $s\ra 1$ of a certain (smooth) function.  We know this
limit is an integer.  But, for example, for the rank $4$ curve 
\begin{equation}\label{eqn:rank4}
  y^2 +xy = x^3 - x^2 - 79x + 289
\end{equation}
of conductor 234446
nobody has succeeded in proving that this integer limit is $4$.  (We
can prove that the limit is either $2$ or $4$ by using 
the functionality equation \eqref{eqn:functional} to show that
the order of vanishing is even, then verifying by computation
that $L^{(4)}(E,1)= 214.65233\ldots \neq 0$.)

Using the definition of derivative, we approximate 
$(s-1)\frac{L'(s)}{L(s)}$ as follows.  For $|s-1|$ small, we have
\begin{align*}
(s-1)\frac{L'(s)}{L(s)} &= 
    \frac{s-1}{L(s)}\cdot \lim_{h\ra 0}  \frac{L(s+h) - L(s)}{h}\\
    &\approx \frac{s-1}{L(s)}\cdot
             \frac{L(s+(s-1)^2) - L(s)}{(s-1)^2}\\
    &= \frac{L(s^2 - s+1) - L(s)}{(s-1)L(s)}
\end{align*}

In fact, we have
$$
  \lim_{s\ra 1}\, (s-1)\cdot \frac{L'(s)}{L(s)} 
        = \lim_{s\ra 1} \frac{L(s^2 - s+1) - L(s)}{(s-1)L(s)}.
$$

We can use this formula in \sage to 
``approximate''~$r$.  First we start
with a curve of rank $2$.
\begin{verbatim}
sage: e = EllipticCurve('389a'); e.rank()
2
sage: L = e.Lseries_dokchitser()
sage: def r(e,s): L1=L(s); L2=L(s^2-s+1); return (L2-L1)/((s-1)*L1)
sage: r(e,1.01)
2.00413534247395
sage: r(e,1.001)
2.00043133754756
sage: r(e,1.00001)
2.00000433133371
\end{verbatim}

Next consider the curve $y^2 +xy = x^3 - x^2 - 79x + 289$
of rank~$4$:
\begin{verbatim}
sage: e =  EllipticCurve([1, -1, 0, -79, 289])
sage: e.rank()
4
sage: L = e.Lseries_dokchitser(100)
sage: def r(e,s): L1=L(s); L2=L(s^2-s+1); return (L2-L1)/((s-1)*L1)
sage: R = RealField(100)
sage: r(e,R('1.01'))
4.0212949184444018810727106489
sage: r(e,R('1.001'))
4.0022223745190806421850637523
sage: r(e,R('1.00001'))
4.0000223250026401574120263050
sage: r(e,R('1.000001'))
4.0000022325922257758141597819
\end{verbatim}

It certainly looks like $\lim_{s\ra 1} r(s) = 4$.  We know that
$\lim_{s\ra 1} r(s)\in\Z$, and if only there were a good way to bound
the error we could conclude that the limit is~$4$.  But this has
stumped people for years, and probably it is nearly impossible without
a deep result that somehow interprets $L''(E,1)$ in a completely
different way.

\newpage
\section{The $p$-adic $\cL$-series}\label{sec:padicldefn}
Fix\footnote{This section is based on correspondence with Robert
  Pollack and Koopa Koo.} an elliptic curve $E$ defined over $\QQ$.
We say a prime $p$ is a prime of \defn{good ordinary reduction} for
$E$ if $p\nmid N_E$ and $a_p \not\equiv 0 \pmod{p}$.  The Hasse bound,
i.e., that $|a_p|< 2\sqrt{p}$ on implies that if $p\geq 5$ then
ordinary at $p$ is the same as $a_p\neq 0$.

In this section, we define for each odd prime number $p$ of good
ordinary reduction for $E$ a $p$-adic $L$-function $L_p(E,T)$.  This
is a $p$-adic analogue of the complex $L$-function 
$L(E,s)$ about which there are similar analogue of the
BSD conjecture.

\subsection{Hensel's lemma and the Teichmuller lift}\label{sec:padicprelim}
The following standard lemma is proved by Newton iteration.
\begin{lemma}[Hensel]\label{lem:hensel}
If $f \in \ZZ_p[x]$ is a polynomial and $\beta \in \ZZ/p\ZZ$
is a multiplicity one root of $\overline{f}$, then there is a unique
lift of $\beta$ to a root of $f$.
\end{lemma}%nice ref -- Koblitz's book which says this first appeared in Lang's thesis in 1952. 
For example, consider the polynomial $f(x) = x^{p-1} - 1$.  By
Fermat's little theorem, it has $p-1$ distinct roots in $\ZZ/p\ZZ$,
so by Lemma~\ref{lem:hensel} there are $p-1$ roots of $f(x)$
in $\ZZ_p$, i.e., all the $p-1$st roots of unity are elements of
$\ZZ_p$.  The \defn{Teichmuller lift} is the map that sends
any $\beta \in (\ZZ/p\ZZ)^*$ to the unique $(p-1)$st root of unity
in $\ZZ_p^*$ that reduces to it. 

The \defn{Teichmuller character} is the homomorphism
$$
  \tau: \ZZ_p^* \to \ZZ_p^*
$$
obtained by first reducing modulo $p$, then sending an element
to its Teichmuller lift. 
The \defn{$1$-unit projection} character is the homomorphism
$$
  \langle\, \bullet\, \rangle: \ZZ_p^* \to 1 + p\ZZ_p
$$
given by
$$
   \langle x \rangle = \frac{x}{\tau(x)}.
$$

\subsection{Modular Symbol and Measures}\label{sec:modsymmeasure}
Let 
$$
  f_E(z) = \sum_{n=1}^{\infty} a_n e^{2\pi i n z} \in S_2(\Gamma_0(N))
$$
be the modular form associated to $E$, which is a holomorphic
function on the extended upper half plane $\h\union\Q\union\{\infty\}$.  
Let 
$$
  \Omega_E = \int_{E(\RR)} \frac{dx}{2y + \ua_1 x + \ua_3} \in \RR
$$ 
be the real period
associated to a minimal Weierstrass equation 
$$
  y^2 + \ua_1 xy + \ua_3 y = x^3 +\ua_2 x^2 + \ua_4 x + \ua_6
$$
for $E$.

The \defn{plus modular symbol map} associated
to the elliptic curve $E$ is the map $\QQ \to \QQ$
given by sending $r\in \QQ$ to 
$$
  [r] = [r]_E=
\frac{2 \pi i }{\Omega_E} 
\left( \int_r^{i\infty} f_E(z) dz + \int_{-r}^{i\infty} f_E(z) dz
\right).
$$
\begin{question}\label{quest:bound}
Let $E$ vary over all elliptic curve over $\QQ$
and $r$ over all rational numbers.  Is the set
of denominators of the rational numbers $[r]_E$ 
bounded?  Thoughts: For a given curve $E$, the
denominators are bounded by the order of the image
in $E(\Qbar)$ of the cuspidal subgroup of 
$J_0(N)(\Qbar)$.  It is likely one can show that
if a prime $\ell$ divides the order of the image
of this subgroup, then $E$ admits a rational $\ell$-isogeny.
Mazur's theorem would then prove that the set of such
$\ell$ is bounded, which would imply a ``yes'' answer
to this question.  Also, for any particular curve $E$,
one can compute the cuspidal subgroup precisely, and
hence bound the denominators of $[r]_E$.
\end{question}

Let $a_p$ be the $p$th Fourier coefficient of $E$ and note
that the polynomial
$$
  x^2 - a_p x + p \con x(x-a_p) \pmod{p}
$$
has distinct roots because $p$ is an ordinary prime. 
Let $\alpha$ be the root of $x^2-a_px+p$ with $|\alpha|_p = 1$,
i.e., the lift of the root $a_p$ modulo $p$, which exists by
Lemma~\ref{lem:hensel}.

Define a \defn{measure on $\ZZ_p^*$} by 
$$
\mu_E(a+p^n \ZZ_p) = 
\frac{1}{\alpha^n} \left[\frac{a}{p^n}\right] -\frac{1}{\alpha^{n+1}} \left[\frac{a}{p^{n-1}}\right].
$$
That $\mu_E$ is a measure follows from the formula
for the action of Hecke operators on modular symbols
and that $f_E$ is a Hecke eigenform.  We will not
prove this here\footnote{Add proof or good reference.}.

\subsection{The $p$-Adic $L$-function}
Define the $p$-adic $L$-function as a function on characters
$$
   \chi \in \Hom(\ZZ_p^*,\CC_p^*)
$$
as follows.
Send a character $\chi$ to
$$
  L_p(E,\chi) = \int_{\ZZ_p^*} \chi\,\, d \mu_E.
$$
We will later make the integral on the right more
precise, as a limit of Riemann sums (see Section~\ref{sec:computepadicl}).


\begin{remark}\label{rm:interp}
For any Dirichlet character $\chi:\Z/n\Z\to\C$, let
$L(E,\chi,s)$ be the entire $L$-function defined by
the Dirichlet series 
$$
   \sum_{n=1}^{\infty} \frac{\chi(n) a_n }{n^s}.
$$
The standard \defn{interpolation property} of $L_p$ is that for
any primitive
Dirichlet character $\chi$ of conductor $p^n$ (for any $n$), we
have
\begin{equation}\label{eq:interp}
L_p(E,\chi) = 
\begin{cases}
p^{n} \cdot g(\chi) \cdot L(E,\bar{\chi},1)/\Omega_E
& \text{ for $\chi\neq 1$,}\\
(1-\alpha^{-1})^2 L(E,1)/\Omega_E & \text{ if $\chi = 1$},
\end{cases}
\end{equation}
where $g(\chi)$ is the Gauss sum:
$$
 g(\chi) =  \sum_{a \mod p^n} \chi(a) e^{\frac{2\pi i a}{p^n}}.
$$
Note, in particular, that $L(E,1)\neq 0$ if and only
if $L_p(E,1)\neq 0$.
\end{remark}


In order to obtain a Taylor series attached
to $L_p$, we view $L_p$ as a $p$-adic analytic function on 
the open disk
$$
  D = \{u \in \C_p \, : \, |u-1|_p < 1 \},
$$
as follows.   We have that $\gamma=1+p$ is a topological generator 
for $1+p\ZZ_p$.  For any $u\in D$, let $\psi_u:1+p\ZZ_p \to \CC_p^*$
be the character given by sending $\gamma$ to $u$ and
extending by using the group law and continuity.
Extend $\psi_u$ to a character $\chi_u:\ZZ_p^* \to \CC_p^*$
by letting $\chi_u(x) = \psi_u(\langle x \rangle)$.
Finally, overloading notation, let
$$
  L_p(E,u) = L_p(E,\chi_u).
$$


\begin{theorem}\label{thm:padiclseries}
The function $L_p(E,u)$ is a $p$-adic analytic function 
on $D$ with  Taylor series
about $u=1$ in the variable $T$
$$
   \cL_p(E,T) \in \QQ_p[[T]].
$$
that converges on $\{z \in \CC_p\, :\,  |z|_p < 1\}$.
(Note that $L_p(E,u) = \cL_p(E, u-1)$.)
\end{theorem}
It is $\cL_p(E,T)$ that we will compute explicitly.

\begin{conjecture}[Mazur, Tate, Teitelbaum]\label{conj:mtt}
$$
\ord_{T} \cL_p(E,T) = \rank E(\QQ).
$$
\end{conjecture}

\begin{proposition}
Conjecture~\ref{conj:mtt} is true if $\ord_{T}\cL_p(E,T) \leq 1$.
\end{proposition}
\begin{proof}[Sketch of Proof]
By Remark~\ref{rm:interp}, we have
$\ord_T(\cL_p(E,T)) = 0$ if and only
if $$r_{E,\an} = \ord_{s=1} L(E,s) = 0.$$  Since 
the BSD rank conjecture (Conjecture~\ref{conj:bsdrank}) is a theorem when
$r_{E,\an} = 0$, Conjecture~\ref{conj:mtt}
is also known under the hypothesis
that $\ord_T(\cL_p(E,T)) = 0$.

Recall that the BSD rank conjecture is also a theorem
when $r_{E,\an}=1$.  It turns out that the same is true
of Conjecture~\ref{conj:mtt} above.
If $\ord_T(\cL_p(E,T)) = 1$, then
a theorem of Perrin-Riou implies that
a certain Heegner point has nonzero $p$-adic
height, hence is non-torsion, so by the
Gross-Zagier theorem $r_{E,\an}=1$.
Kolyvagin's theorem then implies that
$\rank E(\QQ)=1$. 
\end{proof}

\begin{remark}
Mazur, Tate, and Teitelbaum also define an analogue of $\cL_p(E,T)$ for
primes of bad multiplicative reduction and make
a conjecture.  A prime $p$ is \defn{supersingular}
for $E$ if $a_p \con 0 \pmod{p}$; it is a theorem
of Elkies \cite{elkies:supersingular} that for any elliptic curve $E$ there
are infinitely many supersingular primes $p$.
Perrin-Riou, Pollack, Greenberg
and others have studied $\cL_p(E,T)$ at good 
supersingular primes.  More works needs to be done on 
finding a  definition of $\cL_p(E,T)$ when $p$ is a prime
of bad additive reduction for $E$.
\end{remark}


\begin{remark}
A theorem of Rohrlich implies that there is some character
as in \eqref{eq:interp} such that $L(E,\chi,1)\neq 0$,
so $\cL_p(E,T)$ is not identically zero.  Thus $\ord_T \cL_p(T)<\infty$.
\end{remark}

\section{Computing $\cL_p(E,T)$}\label{sec:computepadicl}
Fix notation as in Section~\ref{sec:padicldefn}.  In particular, $E$
is an elliptic curve over $\QQ$, $p$ is an odd prime of good ordinary
reduction for $E$, and $\alpha$ is the root of $x^2-a_px+p$ with
$|\alpha|_p = 1$.

For each integer $n\geq 1$, define a polynomial
$$
P_{n}(T) = 
   \sum_{a=1}^{p-1} \left( \sum_{j=0}^{p^{n-1}-1} \mu_{E}
      \left(\tau(a)(1+p)^j + p^n\ZZ_p\right) \cdot (1+T)^j \right) \in \QQ_p[T].
$$
Recall that $\tau(a)\in\ZZ_p^*$ is the Teichmuller lift of $a$.
\begin{proposition}
We have that the $p$-adic limit of these polynomials is the $p$-adic
$L$-series:
$$
   \lim_{n\to\infty} P_n(T) = \cL_p(E, T).
$$
\end{proposition}


This convergence is coefficient-by-coefficient, in the sense that
if
$P_{n}(T) = \sum_j a_{n,j} T^j$ and 
$\cL_p(E,T) = \sum_j a_j T^j$, then
$$
\lim_{n \to \infty} a_{n,j} = a_j.
$$ 
We now give a proof of this convergence and in doing so obtain an
upper bound for $|a_j - a_{n,j}|$.



For any choice $\zeta_r$ of $p^r$-th root  
of unity in $\C_p$, 
let $\chi_r$ be the $\C_p$-valued 
character of $\Zp^\times$ of order $p^r$ which
factors through $1+p\Zp$ and sends $1+p$ to $\zeta_r$.
%$g(\chi_r)$ is the Gauss sum.  
Note that the conductor of $\chi_r$ is $p^{r+1}$.

\begin{lemma}\label{lem:rval}
Let $\zeta_r$ be a $p^r$-th root of unity with
$1 \leq r \leq n-1$, and let $\chi_r$ be
the corresponding character of order $p^{r+1}$,
as above.  Then
$$
  P_{n}(\zeta_r-1) = \int_{\Zp^\times} \chi_r ~d\mu_{E}
$$
In particular, note that the right hand side does not depend on $n$.
\end{lemma}
\begin{proof}
Writing $\chi=\chi_r$, we have
\begin{align*}
P_{n}(\zeta_r-1) 
&= \sum_{a=1}^{p-1} \sum_{j=0}^{p^{n-1}-1} \mu_{E}\left(\tau(a)(1+p)^j + p^n\Zp\right) \cdot \zeta_r^j \\
&= \sum_{a=1}^{p-1} \sum_{j=0}^{p^{n-1}-1} \mu_{E}\left(\tau(a)(1+p)^j + p^n\Zp\right) \cdot \chi\left((1+p)^j\right) \\
&= \sum_{ b \in (\Z/p^n\Z)^*} \mu_{E}\left(b + p^n\Zp\right) \cdot \chi(b) \\
&= \int_{\Zp^\times} \chi ~d\mu_{E}.
\end{align*}
In the second to the last equality, we use that
$$(\Z/p^n\Z)^* \isom (\Z/p\Z)^* \cross (1+p (\Z/p^n\Z))^*$$
to sum over lifts
of $b\in (\Z/p^n\Z)^*$ of the form $\tau(a)(1+p)^j$, i.e., a Teichmuller lift times
a power of $(1+p)^j$.
In the last equality, we use that $\chi$ has conductor $p^n$,
so is constant on the residue classes modulo $p^n$, i.e.,
the last equality is just the Riemann sums definition of the 
given integral.

%and thus the claim follows from the standard
%interpolation property \eqref{eq:interp} of the $p$-adic $L$-function.  
\end{proof}

%(Note that the conductor of the character $\chi_r$ is $p^{r+1}$.  This
%is part of the indexing problem and is the reason this claim is only
%true under the {\it strict} inequality $r<n$.)

For each positive integer $n$, let $w_n(T) = (1+T)^{p^n}-1$.
\begin{corollary}\label{cor:div}
We have that
$$
w_{n-1}(T) \text{~divides~} P_{n+1}(T) - P_{n}(T).
$$
\end{corollary}
\begin{proof} 
By Lemma~\ref{lem:rval},
$P_{n+1}(T)$ and $P_{n}(T)$ agree on 
$\zeta_j-1$ for $0 \leq j \leq n-1$
and any choice $\zeta_j$ of $p^j$-th root of unity,
so their difference vanishes on every
root of the polynomial
$w_{n-1}(T) = (1+T)^{p^{n-1}} - 1$.
The claimed divisibility follows, since
$w_{n-1}(T)$ has distinct roots. 
\end{proof}

\begin{lemma}\label{lem:minval}
Let 
$f(T) = \sum_j b_j T^j$ and $g(T)=\sum_j c_j T^j$ 
be in $\O[T]$ with $\O$ a finite extension of $\Zp$.  
If $f(T)$ divides $g(T)$, then 
$$
  \ord_p(c_j) \geq \min_{0 \leq i \leq j} \ord_p(b_i).
$$
\end{lemma}
\begin{proof}
We have $f(T)k(T) = g(T)$.
The lemma follows
by using the definition of polynomial
multiplication and the non-archimedean property of $\ord_p$
on each coefficient of $g(T)$.
\end{proof}

As above, let $a_{n,j}$ be the $j$th coefficient of 
the polynomial $P_n(T)$.
Let
$$
   c_n = \max(0, - \min_j \ord_p(a_{n,j}))
$$ 
so that $p^{c_n} P_{n}(T) \in \Zp[T]$, i.e., $c_n$
is the smallest power of $p$ that clears the denominator.
Note that $c_n$ is an integer since $a_{n,j} \in \QQ$.
{\em Probably if $E[p]$ is irreducible then $c_n=0$ -- see
Question~\ref{quest:bound}.}
Also, for any $j>0$,
let
$$
  e_{n,j} = \min_{1 \leq i \leq j} \ord_p \binom{p^n}{i}.
$$
be the min of the valuations of the coefficients of $w_{n}(T)$,
as in Lemma~\ref{lem:minval}.

\begin{proposition}\label{prop:padicerr}
For all $n\geq 0$, we have
$
  a_{n+1,0} = a_{n,0},
$
and for $j>0$,
$$
\ord_p(a_{n+1,j} - a_{n,j}) \geq  e_{n-1,j} - \max(c_{n}, c_{n+1}).
$$
\end{proposition}
\begin{proof}
Let $c = \max(c_{n},c_{n+1})$.
The divisibility of Corollary~\ref{cor:div} implies that there
is a polynomial $h(T)\in \Qp[T]$ with 
$$
w_{n-1}(T) \cdot p^{c} h(T) = p^{c} P_{n+1}(T) - p^{c}P_{n}(T)
$$
and thus (by Gauss' lemma) $p^{c} h(T) \in \Zp[T]$ 
since the right hand side of the equation is integral and $w_{n-1}(T)$ 
is a primitive polynomial.
Applying Lemma~\ref{lem:minval} and renormalizing by $p^{c}$
gives the result.
\end{proof}

%If $p$ is an ordinary prime, then I guess $c_n = 0$ for all $n$
%(Christian should know this better than me -- maybe there is a problem
%with a denominator bounded independent of $n$ when the Galois
%representation isn't surjective.) For $p$ supersingular, $c_n =
%\frac{n+1}{2}$.

For $j$ fixed, $e_{n-1,j} - \max(c_{n+1}, c_n)$
goes to infinity as $n$ grows since the $c_k$ are uniformly
bounded (they are bounded by the power of $p$ that
divides the order of the cuspidal subgroup of $E$).
Thus, $\{ a_{n,j} \}$ is a 
Cauchy and Proposition~\ref{prop:padicerr} implies that that
$$
  \ord_p(a_j - a_{n,j}) \geq  e_{n-1,j} - \max(c_{n+1},c_n).
$$

\begin{remark}
Recall that presently there is not a single
example where we can provably show that 
  $\ord_{s=1} L(E,s) \geq 4$.  Amazingly $\ord_{T}
  \cL_p(E,T)$ is ``computable in practice'' because Kato has proved,
using his Euler system in $K_2$, that
  $\rank E(\Q) \leq \ord_{T} \cL_p(E,T)$ by proving a divisibility
predicted by Iwasawa Theory.  Thus computing elements of $E(\Q)$ gives
a provable lower bound, and approximating $\cL_p(E,T)$ using Riemann
sums gives a provable upper bound -- in practice these meet.
\end{remark}



%%% Local Variables: 
%%% mode: latex
%%% TeX-master: "main"
%%% End: 
