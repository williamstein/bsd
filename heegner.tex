\chapter{Heegner Points and Kolyvagin's Euler System}
\section{CM Elliptic Curves}
In this section we state, and in some cases sketch proofs
of, some basic facts about CM elliptic curves. 

If $E$ is an elliptic curve over a field $K$ we let $\End(E/K)$ be the
ring of all endomorphisms of $E$ that are defined over $K$.

\begin{definition}[CM Elliptic Curve] 
An elliptic curve $E$ over a subfield of $\CC$
has \defn{complex multiplication} if $\End(E/\CC)\neq \Z$.
\end{definition}

\begin{remark}
If $E$ is an elliptic curve over $\QQ$, then $\End(E/\QQ) = \Z$.
This is true even if $E$ has complex multiplication,
in which case the complex multiplication must be defined over a bigger
field than $\QQ$.  The reason $\End(E/\QQ) = \Z$ is because
$\End(E/\QQ)$ acts faithfully on the $1$-dimensional
$\QQ$-vector space of invariant holomorphic
differentials on $E$ over $\Q$ and 
$\End(E/\QQ)$ is finitely generated
as a $\ZZ$-module.
\end{remark}

A \defn{complex lattice} $\Lambda \subset \C$ is a subgroup abstractly
isomorphic to $\ZZ\times \ZZ$ such that $\R\Lambda = \C$.
Using the Weirestrass $\wp$-function associated to the lattice
$\Lambda$, one proves that there is a group isomorphism
$$
  \C/\Lambda \isom E_{\Lambda}(\C),
$$
where $E_{\Lambda}$ is an elliptic curve over $\CC$. 
Conversely, if $E$ is any elliptic curve over $\CC$,
then there is a lattice $\Lambda$ such that $E = E_{\Lambda}$. 
Explicitly, if $\omega_E$ is an invariant differential
we may take $\Lambda$ to be
the lattice of all periods $\int_{\gamma} \omega_E\in \CC$, where 
$\gamma$ runs through the integral homology $\H_1(E(\CC),\Z)$.

\begin{proposition}\label{prop:cend}
Let $\Lambda_1$ and $\Lambda_2$ be complex lattices.  Then
$$
  \Hom(\CC/\Lambda_1, \CC/\Lambda_2) 
  \isom \{ \alpha \in \CC : \alpha \Lambda_1 \subset \Lambda_2\},
$$
where the homomorphisms on the left side are as elliptic 
curves over $\CC$.  Moreover, the complex number $\alpha \in \CC$
corresponds to the homomorphism $[\alpha]$ 
induced by multiplication by $\alpha$,
and the kernel of $[\alpha]$ is isomorphic to $\Lambda_2 / (\alpha \Lambda_1)$.
\end{proposition}

\begin{corollary}\label{cor:homothety}
If $\alpha$ is any nonzero complex number and $\Lambda$ 
is a lattice, then
$\CC/\Lambda\isom \CC/(\alpha\Lambda)$.
\end{corollary}
\begin{proof}
Since multiplication by $\alpha$ sends $\Lambda$ into
$\alpha\Lambda$, Proposition~\ref{prop:cend} implies
that $\alpha$ defines a homomorphism with $0$ kernel,
hence an isomorphism. 
\end{proof}

Now suppose $E/\CC$ is a CM elliptic curve, and let
$\Lambda$ be a lattice such that $E\isom E_{\Lambda}$.
Then 
$$
  \End(E/\CC) \isom \{\alpha \in \CC : \alpha \Lambda \subset \Lambda\}.
$$
\begin{proposition}
Let $E=E_{\Lambda}$ be a CM elliptic curve.
Then there is a complex number $\omega$ and a quadratic
imaginary field such $K$ that 
$$\omega\Lambda\subset \O_K,$$
where $\O_K$ is the ring of integers of $K$.
Moreover, $\End(E/\C)$ is an order (=subring of rank $2$) of $\O_K$.
\end{proposition}
\begin{proof}
Write $\Lambda = \ZZ\omega_1 \oplus \ZZ\omega_2$.
By Corollary~\ref{cor:homothety}, we have
$E_{\Lambda} \isom E_{\omega_1^{-1}\Lambda}$,
so we may assume that $\omega_1 = 1$, i.e., 
that $\Lambda = \ZZ + \beta \ZZ$ for some $\beta\in \CC$.
To complete the proof, we will show that $\omega \Lambda\subset \O_K$
for some quadratic imaginary field $K$ and complex number $\omega$.

By our hypothesis that $E$ is CM
there is a complex number $\alpha \not\in\ZZ$
such that $\alpha\Lambda \subset \Lambda$. 
Fixing a basis for $\Lambda$, we see that 
$\alpha$ acts on $\Lambda$ via a $2\times 2$
integral matrix, so satisfies a quadratic equation.
Thus $\alpha$ is an algebraic integer of degree $2$.
In particular,
there are integers $a,b,c,d$ such that
$$ 
  \alpha 1 = a + b\beta,\qquad\text{and}\qquad
  \alpha \beta = c + d\beta.
$$
Since $\alpha\not\in\Z$, the first equation above 
implies that $\beta \in \QQ(\alpha)$, so since
$\beta\not\in\QQ$, $\QQ(\beta)=\QQ(\alpha)$.
Note that $\beta\not\in\RR$ since $\Lambda$ is a lattice
with basis $1$ and $\beta$,
so $K=\QQ(\beta)$ is a quadratic imaginary field.
Thus the ring $\End(E/\C)$ generated by all such $\alpha$
is an order in the ring $\O_K$ of integers
of an imaginary quadratic field. 
Finally, since $\beta\in K$, there is a complex
number $\omega$ such that $\omega (\Z + \Z\beta) \subset \O_K$,
where $\omega$ is chosen so that $\omega\beta\in\O_K$.
\end{proof}


\subsection{The Set of CM Elliptic Curves with Given CM}
\begin{definition}[Fractional Ideal]
A \defn{fractional ideal} $\a$ of a number field $K$
is an $\O_K$-submodule of $K$ that is isomorphic
to $\Z^{[K:\Q]}$ as an abelian group.  In particular,
$\a$ is nonzero. 
\end{definition}
If $\a$ is a fractional ideal, the \defn{inverse} $\a^{-1}$
of $\a$, which is the set of $x \in K$ such that $x\a \subset \O_K$,
is also a fractional ideal.  Moreover, $\a \a^{-1} = \O_K$.


Fix a quadratic imaginary field $K$.
Let $\Ell(\O_K)$ be the set of $\C$-isomorphism
classes of elliptic curves $E/\C$ with 
$\End(E)\isom \O_K$.  By the above results
we may also view $\Ell(\O_K)$ as the set
of lattices $\Lambda$ with $\End(E_{\Lambda}) \isom \O_K$.

If $\a$ is a fractional $\O_K$ ideal,
then $\a\subset K\subset \C$ is a lattice in $\C$.
For the elliptic curve $E_{\a}$ we have
$$
 \End(E_\a) = \O_K,
$$
because $\a$ is an $\O_K$-module by definition.
Since rescaling a lattice produces an isomorphic
elliptic curve, for any nonzero $c\in K$
the fractional ideals $\a$
and $c\a$ define the same elements of $\Ell(\O_K)$.

 The \defn{class group} $\Cl(\O_K)$ is the group
of fractional ideals modulo principal fractional ideals. 
If $\a$ is a fractional $\O_K$ ideal,
denote by $\oa$ its ideal class in the class
group $\Cl(\O_K)$ of $K$. 
We have a natural map
$$
  \Cl(\O_K) \to \Ell(\O_K),
$$
which sends $\oa$ to $E_{\a}$.

\begin{theorem}
Fix a quadratic imaginary field $K$, and
let $\Lambda$ be a lattice in $\CC$ such
that $E_{\Lambda} \in \Ell(\O_K)$.  Let $\a$
and $\b$ be nonzero fractional $\O_K$-ideals.
Then 
\begin{enumerate}
\item $\a \Lambda$ is a lattice in $\C$,
\item We have $\End(E_{\a \Lambda}) \isom \O_K$.
\item We have $E_{\a\Lambda}\isom E_{\b\Lambda}$
if and only if $\oa = \ob$.
\end{enumerate}
Thus there is a well-defined action of $\Cl(\O_K)$
on $\Ell(\O_K)$ given by 
$$
  \oa E_{\Lambda} = E_{\oa^{-1} \Lambda}.
$$
\end{theorem}


\begin{theorem}
The action of $\Cl(\O_K)$ on $\Ell(\O_K)$
is simply transitive.
\end{theorem}

\begin{example}
Let $K=\Q(\sqrt{-23})$.  Then the class number $h_K$ is $3$.
An elliptic curve with CM by $\O_K$ is $\C/(\Z+(1+\sqrt{-23})/2 \Z)$,
and one can obtain the other two elements of $\Ell(\O_K)$ 
by multiplying
the lattice $\Z+(1+\sqrt{-23})/2 \Z$ by two representative ideal
classes for $\Cl(\O_K)$.
\end{example}

\subsection{Class Field Theory}

Class field theory makes sense for arbitrary number fields, but for
simplicity in this section and because it is all that is needed for
our application to the BSD conjecture, we assume henceforth that $K$
is a totally imaginary number field, i.e., one with no real
embeddings.

%One of the main theorems in algebraic number theory 
%is that the class group $\Cl(\O_K)$ is a finite abelian group.

Let $L/K$ be a finite abelian extension of number fields, and let $\a$
be any unramified prime ideal in $\O_K$.  Let $\b$ be an prime of
$\O_L$ over $\a$ and consider the extension $k_\b = \O_L/\b$ of the
finite field $k_\a=\O_K/\a$.  There is an element $\overline{\sigma}
\in \Gal(k_\b/k_\a)$ that acts via $q$th powering on $k_\b$, where
$q=\#k_\a$.  A basic fact one proves in algebraic number theory is
that there is an element $\sigma \in \Gal(L/K)$ that acts as
$\overline{\sigma}$ on $\O_L/\b$; moreover, replacing $\b$ by a
different ideal over $\a$ just changes $\sigma$ by conjugation. Since
$\Gal(L/K)$ is abelian it follows that $\sigma$ is uniquely determined
by $\a$.  The association $\a \mapsto \sigma = [\a, L/K]$ is called
the \defn{Artin reciprocity map}.

\begin{exercise}
Prove that if an unramified prime $\p$ of $K$ splits completely
in an abelian exension $L/K$, then $[\p,L/K] = 1$.
\end{exercise}

Let $\c$ be an integral
ideal divisible by all primes of $K$ that ramify in $L$, and let
$I(\c)$ be the group of fractional ideals that are coprime to $\c$.
Then the reciprocity map extends to a map 
$$
  I(\c) \to \Gal(L/K)\qquad \quad a\mapsto [\a,L/K]
$$
Let 
$$
 P(\c) = \{(\alpha) : \alpha \in K^*,\quad \alpha\con 1 \pmod{\c}\}.
$$
Here $\alpha \con 1\pmod{\c}$ means that $\ord_\p(\alpha-1) \geq \ord_p(\c)$
for each prime divisor $\p\mid \c$.

\begin{definition}[Conductor of Extension]
  The \defn{conductor} of an abelian extension $L/K$ is the 
largest (nonzero) integral ideal $\c=\c_{L/K}$ of
  $\O_K$ such that $[(\alpha),L/K] = 1$ for all $\alpha\in K^*$ such
  that $\alpha \con 1\pmod{\c}$.
\end{definition}

\begin{proposition}
The conductor of $L/K$ exists. 
\end{proposition} 

If $\c=\c_{L/K}$ is the conductor of $L/K$ then Artin reciprocity induces
a group homomorphism 
$$
  I(\c)/P(\c) \to \Gal(L/K).
$$

\begin{definition}[Ray Class Field]\label{defn:rcf}
Let $\c$ be a nonzero integral ideal of $\O_K$.  A
\defn{ray class field} associated to $\c$
is a finite abelian extension $K_{\c}$ of
$K$ such that whenever $L/K$ is an abelian
extension such that $\c_{L/K} \mid \c$,
then $L\subset K_\c$. 
\end{definition}

\begin{theorem}[Existence Theorem of Class Field Theory]
Given any nonzero integral ideal $\c$ of $\O_K$
there exists a unique ray class field $K_\c$
associated to $\c$, and the conductor of $K_\c$
divides $\c$.  
\end{theorem}


\begin{theorem}[Reciprocity Law of Class Field Theory]\label{thm:cft}
Let $L/K$ be a finite abelian extension.
\begin{enumerate}
\item The Artin map is a surjective homomorphism
$I(\c_{L/K}) \to \Gal(L/K)$.
\item The kernel of the Artin map
is $N_{L/K}(I_L) \cdot P(\c_{L/K})$, where $N_{L/K}(I_L)$
is the group of norms from $L$ to $K$ of the fractional
ideals of $L$.
\end{enumerate}
\end{theorem}

\begin{definition}[Hilbert Class Field]
The \defn{Hilbert class field} of a number field $K$ is the
maximal unramified abelian extension of $K$.
\end{definition}

In particular, since the Hilbert class field is unramified
over $K$, we have:
\begin{theorem}
Let $K$ be a number field and let $H$ be the Hilbert
class field of $K$.
The Artin reciprocity map induces an isomorphism
$$
  \Cl(\O_K) \xrightarrow{\,\,\isom\,\,} \Gal(H/K).
$$
\end{theorem}


\subsection{The Field of Definition of CM Elliptic Curves}\label{sec:cmfield}
\begin{theorem}\label{thm:cmaction}
Let $F$ be an elliptic curve over $\CC$
with CM by $\O_K$, where $K$ is a quadratic
imaginary field.  Let $H$ be the Hilbert
Class Field of $K$.  
\begin{enumerate}
\item There is an elliptic
curve $E$ defined over $K$ such that $F\isom E_{\CC}$.
\item The $\Gal(H/K)$-conjugates
of $E$ are representative elements for $\Ell(\O_K)$.
\item If $\sigma \in \Gal(H/K)$ corresponds via
Artin reciprocity to $\oa\in \Cl(\O_K)$, then
$$
  E^{\sigma} = \oa E.
$$
\end{enumerate}
\end{theorem}

Theorem~\ref{thm:cmaction} generalizes in a natural way to the
more general situation in which $\O_K$ is replaced by an order $\O_f =
\Z + f\O_K \subset\O_K$.  Then the Hilbert class field is replaced by
the ray class field $K_f$, which is a finite abelian extension of $H$
that is unramified outside $f$ (see Definition~\ref{defn:rcf} above). 
There is an elliptic curve $E$ defined
over $K_f$ whose endomorphism ring is $\O_f$, and the set of
$\Gal(K_f/K)$-conjugates of $E$ forms a set of representatives for
$\Ell(\O_f)$.  Moreover, the group
$I(\c_{L/K})/(N\cdot P(\c_{L/K}))$
of Theorem~\ref{thm:cft}
acts simply transitively on
$\Ell(\O_f)$, and the action of $\Gal(K_f/K)$ on the set of
conjugates of $E$ is consistent with the Artin reciprocity map.


\newpage
\section{Heegner Points}
Let $E$ be an elliptic curve defined over $\QQ$ with conductor $N$,
and fix a modular parametrization $\pi_E:X_0(N) \to E$.  

Let $K$ be a quadratic imaginary field such that the primes dividing
$N$ are all unramified and split in $K$.  For simplicity, we will also
assume that $K\neq \Q(i), \Q(\sqrt{-3})$.  Let $\cN$ be an integral
ideal of $\O_K$ such that $\O_K/\cN \isom \Z/N\Z$.
Then $\C/\O_K$ and $\C/\cN^{-1}$ define two elliptic curves
over $\CC$, and since $\O_K \subset \cN^{-1}$, there is 
a natural map 
\begin{equation}\label{eqn:okcn}
 \C/\O_K \to \C/\cN^{-1}.
\end{equation}
By Proposition~\ref{prop:cend} the kernel of this map
is 
$$
  \cN^{-1}/\O_K \isom \O_K/\cN\isom \Z/N\Z.
$$
\begin{exercise}
Prove that there is an isomorphism
$\cN^{-1}/\O_K \isom \O_K/\cN$
of finite abelian group.
\end{exercise}
The modular curve $X_0(N)$ parametrizes isomorphism
classes of pairs $(F,\phi)$, where $\phi$ is an isogeny
with kernel cyclic of order $N$.  Thus $\C/\O_K$
and the isogeny \eqref{eqn:okcn} define an element
$x_1 \in X_0(N)(\CC)$.   The discussion of Section~\ref{sec:cmfield}
along with properties of modular curves proves the following
proposition.
\begin{proposition}
We have
$$
  x_1 \in X_0(N)(H),
$$ 
where $H$ is the Hilbert class field of $K$.
\end{proposition}

\begin{definition}[Heegner point]
The \defn{Heegner point} associated to $K$ is 
$$y_K = \Tr_{H/K}(\pi_E(x_1)) \in E(K).$$
\end{definition}

More generally, for any integer $n$, let 
$\O_n = \Z + n\O_K$ be the order in $\O_K$ of
index $n$.  Then $\cN_n = \cN \cap \O_n$ satisfies
$\O_n/\cN_n \isom \Z/N\Z$, and the pair
$$
 (\C/\O_n,\,\, \C/\O_n \to \C/\cN_n^{-1})
$$
defines a point $x_n \in X_0(N)(K_n)$,
where $K_n$ is the ray class field of 
conductor~$n$ over $K$.

\begin{definition}[Heegner point of conductor $n$]
The Heegner point of conductor $n$ is
$$
  y_n = \pi_E(x_n) \in E(K_n).
$$
\end{definition}

%\subsection{An Example}
%We compute a Heegner point on the rank $1$ elliptic curve 
%37a given by the equation $y^2 + y = x^3 - x$.  See
%\cite{zagier:modular} for much much more about this example.

\section{Computing Heegner Points}
[[This section will be my take on what's in Cohen's book
and Watkins paper, hopefully generalized to compute Heegner
points over ring class fields (?).]]

\section{Kolyvagin's Euler System}
\subsection{Kolyvagin's Cohomology Classes}
In this section we define Kolyvagin's cohomology classes.  Later we
will explain the properties that these classes have, and eventually
use them to sketch a proof of finiteness of Shafarevich-Tate
groups of certain elliptic curves.

We will use, when possible, similar notation to the notation
Kolyvagin uses in his papers (e.g.,
\cite{kolyvagin:structure_of_selmer}).  If $A$ is an abelian group let
$A/M = A/(MA)$.  Kolyvagin writes $A_M$ for the $M$-torsion subgroup,
but we will instead write $A[M]$ for this group.

Let $E$ be an elliptic curve over $\QQ$ with no constraint
on the rank of $E$.
Fix a modular parametrization $\pi:X_0(N)\to E$, where $N$
is the conductor of $E$.
Let $K$ be a quadratic imaginary field with discriminant $D$
that satisfies the Heegner hypothesis for $E$, so each
prime dividing $N$ splits in $K$,
and assume for simplicity that $D\neq -3, -4$. 


Let $\O_K$ be the ring of integer of $K$. Since $K$ satisfies the
Heegner hypothesis, there is an ideal $\cN$ in $\O_K$ such that
$\O_K/\cN$ is cyclic of order~$N$.  For any positive integer
$\lambda$, let $K_{\lambda}$ be the ray class field of $K$
associated to the conductor $\lambda$ (see Definition~\ref{defn:rcf}).
Recall that $K_{\lambda}$ is an abelian extension of~$K$
that is unramified outside~$\lambda$, whose existence is
guaranteed by class field theory.
Let $\O_{\lambda} = \Z + \lambda \O_K$ be the order in $\O_K$
of conductor $\lambda$, and let $\cN_{\lambda} = \cN \cap \O_{\lambda}$.
Let 
$$
  z_{\lambda} = 
[(\CC / \O_{\lambda}, \cN_{\lambda}^{-1} / \O_{\lambda})]
  = X_0(N)(K_{\lambda})
$$
be the Heegner point associated to $\lambda$. 
Also, let
$$
  y_{\lambda} = \pi(z_\lambda) \in E(K_{\lambda})
$$
be the image of the Heegner point on the curve $E$.

Let $R=\End(E/\C)$, and
let $B(E)$ be the set of primes $\ell \geq 3$ in $\ZZ$
that do not divide the discriminant of $R$ and
are such that the image of the representation 
$$
  \rho_{E,\ell} : \Gal(\Qbar/\Q) \to \Aut(\Tate_{\ell}(E))
$$
contains $\Aut_R(\Tate_{\ell}(E))$, 
where $\Aut_R(\Tate_{\ell}(E))$
is the set of automorphisms that commute with the
action of $R$ on $\Tate_{\ell}(E)$.   
Note that if $\ell \geq 5$ the condition that
$\rho_{E,\ell}$ is surjective is equivalent to the simpler condition
that 
$$
  \rhobar_{E,\ell} : \Gal(\Qbar/\Q) \to \Aut_{R}(E[\ell])
$$
is surjective.
The set $B(E)$ contains all but finitely many primes, by theorems of Serre
\cite{serre:propgal}, Mazur \cite{mazur:rational}, and CM theory,
and one can compute $B(E)$.
\begin{verbatim}
sage: E = EllipticCurve('11a')
sage: E.non_surjective()
[(5, '5-torsion')]
sage: E = EllipticCurve('389a')
sage: E.non_surjective()
[]
\end{verbatim}

Fix a prime $\ell\in B(E)$.  We next introduce some 
very useful notation.  Let $\Lambda^1$ denote the set of all
primes $p\in\ZZ$ such that $p\nmid N$, $p$ remains prime in $\O_K$,
and for which
$$
  n(p) = \ord_{\ell}(\gcd(p+1, a_p)) \geq 1.
$$
For any positive integer $r$, let $\Lambda^r$
denote the set of all products of $r$ distinct
primes in $\Lambda^1$; by definition $\Lambda^0 = \{ 1\}$. 
Finally, let
$$
  \Lambda  = \bigcup_{r\geq 0} \Lambda^r.
$$
For any $r > 0$ and $\lambda \in \Lambda^r$, let
$$
  n(\lambda) = \min_{p\mid \lambda} n(p) 
$$
be the ``worst'' of all the powers of $p$ that divide $\gcd(p+1,a_p)$. 
If $\lambda = 1$, set $n(\lambda)=+\infty$.

Fix an element $\lambda \in \Lambda$, with $\lambda\neq 1$, and
consider the $\ell$-power 
$$
 M = M_{\lambda} = \ell^{n(\lambda)}.
$$
Recall from Section~\ref{sec:kummer} that we associate to the 
short exact sequence
$$
  0 \to E[M] \to E \xrightarrow{[M]} E \to 0
$$
an exact sequence 
$$
 0 \to E(K)/M  \to \H^1(K,E[M]) \to H^1(K,E)[M]\to 0.
$$
Our immediate goal is to construct an {\em interesting} cohomology class
$$
  c_{\lambda} \in \H^1(K,E[M]).
$$

If $L/K$ is any Galois extension, we have
(see Section~\ref{sec:infres} for most of this)
an exact sequence
\begin{equation}\label{eq:infresstop}
  0 \to \H^1(L/K, E[M](L)) \to \H^1(K, E[M]) \to \H^1(L,E[M])^{\Gal(L/K)} \to 0.
\end{equation}

\begin{lemma}\label{lem:notor}
We have $E[M](K_{\lambda}) = 0$.
\end{lemma}
\begin{proof}
For simplicity we prove the statement only in the non-CM case. 
The integer $M$ is a power of a prime $\ell$, 
so it suffices to show that $E[\ell](K_{\lambda}) = 0$.
Since  $\ell\in B(E)$ the Galois representation
$$
  \rhobar_{E, \ell}: G_\Q \to \GL_2(\F_\ell)
$$
is surjective. The group $\GL_2(\F_\ell)$ acts transitively
on $(\F_\ell)^2$, so the $G_\Q$ orbit of any nonzero
point in $E[\ell](\Qbar)$ is equal to the set of all nonzero 
points in  $E[\ell](\Qbar)$.
By class field theory, the extension $K_{\lambda}$ of
$\Q$ is Galois, so if $E[\ell](K_\lambda)$ is nonzero,
then it is equal to $E[\ell](\Qbar)$.  Using
properties of the Weil pairing, we see that  the field generated by
the coordinates of the elements of $E[\ell](\Qbar)$ contains
the cyclotomic field $\Q(\zeta_{\ell})$, which is a field
totally ramified at $\ell$.  
 But $K\cap \Q(\zeta_{\ell}) = \Q$,
since $\disc(K)\neq -3, -4$, and $K_{\lambda}$ is ramified
only at primes in $\Lambda^1$ and $\ell\not\in\Lambda^1$. 
We conclude that $K_{\lambda} \cap \QQ(\zeta_{\ell}) = \QQ$,
so we must have $E[\ell](K_\lambda) = 0$.
(Compare \cite[Lem.~4.3]{gross:kolyvagin}.)
\end{proof}

Thus \eqref{eq:infresstop} with $L=K_{\lambda}$ becomes
\begin{equation}\label{eqn:resklambda}
   \H^1(K, E[M]) \xrightarrow{\quad\isom\quad} \H^1(K_{\lambda},E[M])^{G_{\lambda}}
 \end{equation}
where $G_{\lambda} = \Gal(K_{\lambda}/K)$.  Putting this together,
we obtain the following commutative diagram with
exact rows and columns:

$$
\xymatrix{
 0 \ar[r]& {(E(K_{\lambda})/M)^{G_{\lambda}}}  \ar[r]& {\H^1(K_{\lambda},E[M])^{G_{\lambda}}} \ar[r] & \H^1(K_\lambda, E)[M]^{G_\lambda}\\
 0 \ar[r]& {E(K)/M}  \ar[r]\ar@{^(->}[u]& {\H^1(K,E[M])}\ar^{\isom}_{\res}[u]\ar[r]
    & \H^1(K,E)[M] \ar[r]\ar[u]^{\res} & 0\\
  & & & \H^1(K_\lambda/K, E)[M]\ar@{^(->}[u]^{\inf}\\
}
$$

Thus to construct $c_\lambda \in \H^1(K,E[M])$, it suffices
to construct a class $c_{\lambda}' \in \H^1(K_{\lambda},E[M])$
that is invariant under the action of $G_{\lambda}$. 
We will do this by constructing an element of $E(K_{\lambda})$
and using the inclusion
\begin{equation}\label{eqn:incklambda}
   E(K_{\lambda})/M \hra \H^1(K_{\lambda}, E[M]).
 \end{equation}
 In particular, we will construct an element of 
the group $E(K_{\lambda})/M$ that is invariant under
the action of $G_{\lambda}$. 

Recall that $y_\lambda \in E(K_{\lambda})$.  Unfortunately,
there is no reason that the class
$$
  [y_\lambda] \in  E(K_{\lambda})/M
$$
should be invariant under the action of $G_{\lambda}$.
To deal with this problem, Kolyvagin introduced a
{\em new and original idea} which we now explain.


Let $H=K_1$ be the Hilbert class field of $K$.
Write $\lambda = p_1 \cdots p_r$, and for each $p = p_i$ let
$G_{p} = \Gal(K_{p}/K)$ where $K_{p}$ is the ray class
field associated to $p$.  Class field theory implies
that the natural map
$$
   \Gal(K_\lambda/K_1) \isom \to G_{p_1} \cross G_{p_2} \cross \cdots \cross G_{p_r}
$$
is an isomorphism.  Moreover, each group $G_{p_i}$ is cyclic
of order $p_i + 1$.  For each $p=p_i$, let $\sigma_p$ be a fixed
choice of generator of $G_{p}$, and let
$$
  \Tr_{p} = \sum_{\sigma \in G_{p}} \sigma \in \ZZ[G_{p}].
$$
Finally, let $D_{p} \in \ZZ[G_p]$ 
be any solution of the equation
\begin{equation}\label{eqn:dpeq}
  (\sigma_p - 1) \cdot D_p = p  + 1 - \Tr_p.
\end{equation}
For example, Kolyvagin always takes 
$$
 D_p = \sum_{i=1}^p i \sigma_p^i 
= -\sum_{i=1}^{p+1} (\sigma_p^i - 1)/(\sigma_p-1).
$$
Notice that the choice of $D_p$ is well defined up to addition of elements
in $\ZZ \Tr_p$. 
Let
$$ 
  D_\lambda = \prod D_p = D_{p_1} \cdot D_{p_2} \cdot \cdots \cdot D_{p_r}
     \in \ZZ[G_\lambda].
$$

Finally, let $S$ be a set of coset representatives
for $\Gal(K_\lambda/K_1)$ in $G_{\lambda} = \Gal(K_{\lambda}/K)$,
and let 
$$
 J_{\lambda} = \sum_{\sigma \in S} \sigma \in \ZZ[G_{\lambda}].
$$


Let 
$$
   P_\lambda = J_{\lambda} D_\lambda y_\lambda \in E(K_{\lambda}).
$$
Note that if $\lambda=1$, then $K_{\lambda} = K_1$, so 
$$
  P_1 = J_1 y_{\lambda} = \Tr_{K_1/K}(y_\lambda) = y_K \in E(K).
$$

Before proving that we can use $P_{\lambda}$ to define
a cohomology class in $\H^1(K,E[M])$, we state two 
crucial facts about the structure of the Heegner points $y_\lambda$.

\begin{proposition}\label{prop:heegnertrace}
Write $\lambda = p \lambda'$, and let $a_p = a_p(E) = p+1 - \# E(\F_p)$.
\begin{enumerate}
\item We have
$$
  \Tr_p(y_\lambda) = a_p y_{\lambda'}
$$
in $E(K_{\lambda'})$.
\item Each prime factor $\wp_{\lambda}$ of $p$ in $K_{\lambda}$
divides a unique prime $\wp_{\lambda'}$ of $K_{\lambda'}$,
and we have a congruence
$$
   y_{\lambda} \con \Frob(\wp_{\lambda'})(y_{\lambda'}) \pmod{\wp_{\lambda}}.
$$
\end{enumerate}
\end{proposition}
\begin{proof}
See \cite[Prop.~3.7]{gross:kolyvagin}.  The proof uses
a description of the action of Hecke operators on modular curves.
\end{proof}

\begin{proposition}\label{prop:pointfixed}
The class $[P_\lambda]$ of $P_\lambda$
in $E(K_{\lambda})/M$ is fixed by $G_\lambda$.
\end{proposition}
\begin{proof}
We follow the proof of \cite[Prop.~3.6]{gross:kolyvagin}.
It suffices to show that $[D_\lambda y_\lambda]$ is fixed
by $\sigma_p$ for each prime $p\mid \lambda$, since the $\sigma_p$
generate $\Gal(K_{\lambda}/K_1)$, the elements of the set $S$
of coset representatives fix the image 
of $J_{\lambda}$, and $G_\lambda$ is generated
by the $\sigma_p$ and $S$.
Thus we will prove that
$$
  (\sigma_p - 1) D_\lambda y_\lambda \in M E(K_{\lambda})
$$
for each $p\mid \lambda$.

Write $\lambda = p m$.  By \eqref{eqn:dpeq}, we have  in $\ZZ[G_{\lambda}]$
that
$$
  (\sigma_p - 1) D_\lambda = (\sigma_p - 1) D_p D_m = (p + 1 - \Tr_p) D_m,
$$
so using Proposition~\ref{prop:heegnertrace} we have
\begin{align*}
  (\sigma_p - 1) D_\lambda y_\lambda 
  &= (p + 1 - \Tr_p) D_m y_\lambda \\
  &= (p+1) D_m y_\lambda - D_m \Tr_p(y_\lambda)\\
  &= (p+1) D_m y_\lambda - a_p D_m y_{\lambda'}\\
\end{align*}
Since $p\in\Lambda^1$ and $M=\ell^{n(p)}$ and
$n(p) = \min(\ord_\ell(p+1), \ord_\ell(a_p))$, 
we have $M \mid p+1$ and $M\mid a_p$.
Thus $(p+1) D_m y_\lambda \in M E(K_{\lambda})$
and $a_p y_{\lambda'} \in M E(K_{\lambda})$, which
proves the proposition.
\end{proof}


We have now constructed an element of $E(K_\lambda)/M$ 
that is fixed by $G_{\lambda}$.  Via \eqref{eqn:incklambda}
this defines an element $c'_{\lambda} \in \H^1(K_{\lambda}, E[M])$.  
But then using \eqref{eqn:resklambda} we obtain our
sought after class $c_{\lambda} \in \H^1(K, E[M])$.


We will also be interested in the image $d_{\lambda}$
of $c_{\lambda}$ in $\H^1(K,E)[M]$.

\begin{proposition}
If $v$ is archimedean or $v\nmid \lambda$, then 
$$
  \res_v(d_{\lambda}) = 0.
$$
\end{proposition}
\begin{proof}
If $v$ is archimedean we are done, since $K_v = \C$
is algebraically closed.
Otherwise, the class $d_{\lambda}$ splits over $K_{\lambda}$
and $K_{\lambda}$ is unramified at $v$, so
$$
  \res_v(d_\lambda) \in \H^1(K_v^{\unr}/K_v, E).
$$
But the latter group is isomorphic to the component
group of $E$ at $v$, and a theorem of Gross-Zagier
implies that the Heegner point maps to the identity
component.  (See \cite[Prop.~6.2]{gross:kolyvagin} for
more details.)
\end{proof}

\begin{proposition}\label{prop:localorder}
Write $\lambda = p m$ and let $\wp = p\O_K $ be the unique prime ideal
of $K$ dividing $p$.   Let $v$ be a place of $K_m$ that divides $\wp$.
Then the order of
$
  \res_\wp(d_\lambda)
$ is
the same as the order
of 
$$
[P_m] \in E(K_\wp)/M E(K_{\wp}),
$$
where $K_\wp$ denotes the completion of $K$ at $\wp$.
(Note that $\wp$ splits completely in $K_m/K$ by class
field theory, since 
$\wp = p\O_K$ is principal and coprime to $m$, so 
$P_m \in E(K_\wp)$.)
\end{proposition}
\begin{proof}
See \cite[Prop.~6.2]{gross:kolyvagin} for the case $M=\ell$.  The
argument involves standard properties of Galois cohomology
of elliptic curves, some diagram chasing, reduction
modulo a prime, and use of formal groups.
\end{proof}

Next we consider a consequence of Proposition~\ref{prop:localorder}
when $y_K$ is not a torsion point.  Note that $y_K$ nontorsion implies
that $y_K \not \in M E(K)$ for all but finitely many $M$.  Moreover,
the Gross-Zagier theorem implies % todo: giv ref
that $y_K$ is nontorsion if and only if $\ord_{s=1} L(E,s) \leq 1$.

% \begin{lemma}
% Suppose $y_K \in E(K)$ is 
% not divisible by~$\ell$ and let $H=K_1$ be the Hilbert
% class field of $K$.
% Then $y_K$ is not divisible by $\ell$ in $E(H)$ either.
% \end{lemma}
% \begin{proof}
% Suppose $\ell z = y_K$ with $z \in E(H)$. 
% Since $H/K$ is Galois and $z\not\in E(K)$,
% there exists $\sigma \in \Gal(H/K)$
% such that $\sigma(z)\neq z$.  Then the difference
% $\sigma(z) - z \in E(H)[\ell]$ is nonzero, which
% contradicts Lemma~\ref{lem:notor} with $\lambda = 1$.
% \end{proof}

\begin{proposition}\label{prop:nonzeroheegner}
Suppose that $y_K \in E(K)$ is not divisible by $M$.
Then there are infinitely many $p\in\Lambda^1$
such that  $d_{p} \in \H^1(K,E)[M]$ is nonzero.
\end{proposition}
\begin{proof}
This follows from 
Proposition~\ref{prop:localorder} with $m=1$
and the Chebotarev density
theorem.   See e.g., \cite[\S4.1]{stein:index} for a proof.
%By ,
%it suffices to show that $y_K\not\in M E(K_\wp)$
%for infinitely many $p\in \Lambda^1$, where 
%as usual $\wp = p\O_K$.
\end{proof}

\begin{remark}
 See, e.g., \cite{stein:index} for an application of this
idea to a problem raised by Lang and Tate in \cite{lang-tate}.
\end{remark}


\begin{theorem}[Kolyvagin]
Suppose $E$ is a modular elliptic curve over $\QQ$ and
$K$ is a quadratic imaginary field that satisfies
the Heegner hypothesis for $E$ and is such that
$y_K \in E(K)$ is nontorsion.
Then $E(K)$ has rank $1$ and 
$$
  \#\Sha(E/K) \mid b \cdot [E(K):\Z y_K]^2,
$$
where $b$ is a positive integer divisible
only by primes $\ell \in B(E)$ (i.e., for
which the $\ell$-adic representation is
not as surjective as possible).
\end{theorem}
\begin{proof}
See the entire paper \cite{gross:kolyvagin}.
Kolyvagin proves this theorem by 
 bounding $\Sel^{(M)}(E/K)$ for various $M$
using Proposition~\ref{prop:nonzeroheegner}
in conjunction with a careful study of
various pairings coming from Galois cohomology,
the Weil pairing, Tate local daulity, etc.
Since
$$
  0 \to E(K)/M E(K) \to \Sel^{(M)}(E/K) \to \Sha(E/K),
$$
a bound on the Selmer group translates into a
bound on $E(K)$ and $\Sha(E/K)$. 
\end{proof}

After Kolyvagin proved his theorem, independently
 Murty-Murty, Bump-Friedberg-Hoffstein, Waldspurger,
each proved that infinitely many such quadratic imaginary 
$K$ always exists so long as $E$ has analytic rank $0$
or $1$.  Also, Taylor and Wiles proved
that every $E$ over $\Q$ is modular.  Thus we have
the following theorem:
\begin{theorem}
Suppose $E$ is an elliptic curve over $\QQ$ with
$$r_{E,\an} = \ord_{s=1}L(E,s) \leq 1.$$
Then $E(\QQ)$ has rank $r_{E,\an}$, the group $\Sha(E/\Q)$
is finite, and there is an explicit computable
upper bound on $\#\Sha(E/\QQ)$.
\end{theorem}

The author has computed the upper bound of the theorem
for all elliptic curves with conductor up to $1000$ and
$r_{E,\an}\leq 1$.

\subsection{Kolyvagin's Conjectures}
What about curves $E$ with $r_{E,\an}\geq 2$?  Suppose
that $E$ is an elliptic curve over $\Q$ with
$r_{E,\an}\geq 2$.  
In the short paper \cite{kolyvagin:structure_of_selmer},
Kolyvagin states an amazing structure 
theorems for Selmer groups assuming the following unproved conjecture,
which is the appropriate generalization of
the condition that $P_1$ has infinite order. 
\begin{conjecture}[Kolyvagin \cite{kolyvagin:structure_of_selmer}]\label{conj:kolya}
Let $E$ be any elliptic curve over $\QQ$ and fix
a prime $\ell\in B(E)$ and a prime power $M=\ell^n$
of $\ell$.
Then there is at least one cohomology class 
$c_{\lambda} \in H^1(K,E[M])$
that is nonzero.
\end{conjecture}
So far nobody has been able to show that Conjecture~\ref{conj:kolya}
is satisfied by every elliptic curve $E$ over $\Q$, though several
people are currently working hard on this problem (including
Vatsal and Cornut).  Proposition~\ref{prop:nonzeroheegner}
above implies that Conjecture~\ref{conj:kolya} is true for
elliptic curves with $r_{E,\an}\leq 1$.

Kolyvagin also goes on in \cite{kolyvagin:structure_of_selmer}
to give a {\em conjectural construction}
of a subgroup 
$$
 V \subset E(K)/E(K)_{\tor}
$$
for which $\rank(E(\Q)) = \rank(V)$.
Let $\ell$ be an arbitrary prime, i.e., so we do not necessarily
assume $\ell\in B(E)$.  
One can construct cohomology class $c_{\lambda} \in \H^1(K, E[M])$,
so long as $\lambda \in \Lambda^{n + k_0}$, where 
$\ell^{k_0/2} E(\mathbf{K})[\ell^\infty] = 0$, and
$\mathbf{K}$ is the compositum of all class field $K_{\lambda}$
for $\lambda \in \Lambda$.  For any $n\geq 1$,
$k\geq k_0$, and $r\geq 0$, let
$$
  V_{n,k}^r \subset \varinjlim_{m}\H^1(K,E[\ell^m])/E(K)_{\tor}
$$
be the subgroup generated by the images of the 
classes $\tau_{\lambda} = \tau_{\lambda,n}\in\H^1(K,E[\ell^n])$ 
where $\lambda$ runs through $\Lambda_{n+k}^r$.

\begin{conjecture}[Kolyvagin]
Let $E$ be any elliptic curve over $\Q$.  Then
for all prime numbers $\ell$, there exists
an integer $r$ such that for all $k\geq k_0$
there is an $n$ such that $V_{n,k}^r\neq 0$.
\end{conjecture}

Recall that
$$n(p) = \ord_\ell(\gcd(p+1, a_p)) \geq 1$$
and
$$
  n(\lambda) = \min_{p\mid \lambda} n(p).
$$
Let $m'(\lambda)$ be the maximal nonnegative
integer such that $P_{\lambda} \in \ell^{m'(\lambda)} E(K_{\lambda})$.
Let $m(\lambda) = m'(\lambda)$ if $m'(\lambda) < n(\lambda)$,
and $m(\lambda) = \infty$ otherwise. 
For any $r\geq 0$, let
$$
  m_r = \min \{ m(\lambda) : \lambda \in \Lambda^r \},
$$
and let $f$ be the minimal $r$ such that $m_r$
is finite.

\begin{proposition}
We have $f=0$ if and only if $y_K$ has infinite order.
\end{proposition}

Let $SD = \ell^n S$, where 
$$
 S = \varinjlim_{n} \Sel^{(\ell^n)}(K,E[\ell^n]) .
$$
If 
$A$ is a $\Z[1,\sigma]$-module and $\eps = (-1)^{r_{E,\an}-1}$.
then 
$$A^v = \{b \in A : \sigma(b) = (-1)^{v+1} \eps b\}$$

Assuming his conjectures, Kolyvagin deduces that
for every prime number~$\ell$ there exists
integers $k_1$ and $k_2$ such that for
any integer $k\geq k_1$ we have
$$
   \ell^{k_2} SD^{(f+1)}[M] \subset V_{n,k}^f \subset SD^{(f+1)}[M].
$$
Here the exponent of $f+1$ means the $+1$ or $-1$ eigenspace
for the conjugation action.
   

\begin{conjecture}[Kolyvagin]
Let $E$ be any elliptic curve over $\QQ$
and $\ell$ any prime.
There exists $v\in \{0,1\}$ and a subgroup
$$
  V \subset (E(K)/E(K)_{\tors})^{(v)}
$$
such that 
$$
  1 \leq \rank(V) \equiv v\pmod{2}.
$$
Let $a = \rank(V) - 1$.
Then for all sufficiently large $k$ and all $n$,
one has that
$$
  V_{n,k}^a \equiv V \mod{\ell^n (E(K)/E(K)_{\tor})}.
$$
\end{conjecture}

Assuming the above conjecture for all primes $\ell$, the group $V$ is
uniquely determined by the congruence condition in the second part of
the conjecture.  Also, Kolyvagin proves that if the above conjecture
is true, then the rank of $E^v(\Q)$ equals the rank of $V$, 
and that $\Sha(E^v/\Q)[\ell^\infty]$ is finite.  (Here $E^v$ is $E$
or its quadratic twist.)

When $P_1$ has infinite order, the conjecture is true
with $v = 1$ and $V=\Z P_1$.  (I think here $E$ has $r_{E,\an}=0$.)


\newpage
\section{The Gross-Zagier Theorem}

%%% Local Variables: 
%%% mode: latex
%%% TeX-master: "main"
%%% End: 
